\documentclass[a4paper,twoside]{article}

\usepackage{amsmath}
\usepackage[spanish]{babel}
\usepackage[T1]{fontenc}
\usepackage[utf8]{inputenc}
\usepackage[light]{kpfonts}
\usepackage{eulervm}
\usepackage[dvipsnames]{xcolor}
\usepackage{xspace}
\usepackage{natbib}

\usepackage{marginnote}
\usepackage[top=1.5cm,bottom=1.5cm,outer=6cm,inner=4cm,heightrounded,marginparwidth=4cm,%
            marginparsep=1.2cm]{geometry}

\newcommand{\note}[1]{\color{OrangeRed}(#1)\xspace}
\newcommand{\definition}[2]{\marginnote{{\color{teal}#1:} #2}}

\newcommand{\Hip}{\ensuremath{\mathbold{\Theta}}\xspace}
\newcommand{\hip}{\ensuremath{\mathbold{\theta}}\xspace}
\newcommand{\dat}[1][i]{\ensuremath{\{\mathcal{D}_{#1}\}}\xspace}
\newcommand{\pos}[2]{\ensuremath{\pi_N\left(#1\,\middle|\, #2\right)}\xspace}
\newcommand{\pri}[2]{\ensuremath{\pi_0\left(#1\,\middle|\, #2\right)}\xspace}
\newcommand{\lik}[3][]{\ensuremath{\mathcal{L}_{#1}\left(#2\,\middle|\, #3\right)}\xspace}
\newcommand{\pro}[2]{\ensuremath{Pr\left(#1\,\middle|\, #2\right)}\xspace}
\newcommand{\set}[1]{\ensuremath{\left\{#1\right\}}\xspace}

\title{Inferencia en el marco de la Estadística Bayesiana}
\author{Alfredo Mejía-Narváez}
\date{\today}

\begin{document}

\maketitle

\begin{abstract}
%
La estadística representa un conjunto de reglas diseñadas para expresar nuestro grado de
conocimiento dados cierta evidencia y/o conocimiento previo y para avanzar dicho conocimiento a la
luz de nuevas observaciones. Sin embargo, nuestra línea de razonamiento científico se ve forzada a
desviarse un poco de este esquema, que es la forma natural, y la pregunta de interés científico
real: ¿cuál es nuestro grado de conocimiento sobre un evento, dados un conjunto de observaciones y
conocimiento previo o prejuicios? se transforma en otra cosa en la que se supone que la hipótesis
planteada es cierta: dado que tenemos conocimiento absoluto del estado del problema (e.g., el modelo
con el cual describimos los datos con sus estadísticas correspondientes), ¿cuál es la verosimilitud
de los datos bajo cierta hipótesis?. La estadística Bayesiana presenta un esquema que permite
responder naturalmente a la pregunta de interés científico y al mismo tiempo de la manera más
objetiva posible. En este seminario les presentaré el esquema general de la inferencia bayesiana y
algunas de sus aplicaciones.
%
\end{abstract}

%---------------------------------------------------------------------------------------------------

\section*{Introducción: estadística básica}

El concepto de probabilidad ha venido a responder una pregunta fundamental para el ciencia y otras
áreas del conocimiento en general: ¿cuál es el grado de certidumbre que poseemos sobre un evento,
dado conocimiento relevante de fondo?. En este sentido, una vez nuestro grado de certidumbre sobre
un evento queda por sentado también así queda nuestro grado de intertidumbre. Por ejemplo, si
sabemos que existe un $74\%$ de probabilidad de que llueva, sabemos que existe un $26\%$ de que
no. Es en esta \emph{completitud} en la que yace la mayor fortaleza del concepto de probabilidad y
que es y debe ser particularmente apreciado en las ramas del conocimiento científico.

Las operaciones matemáticas sobre las cuales se construyen las bases de una teoría de probabilidad
son bastante simples y están bien definidas. Hoy en día se conocen como las reglas de Cox y
matemáticamente se resumen en dos ecuaciones:
%
\begin{subequations}
\begin{align}
%
1                            &= \pro{\theta}{I} + \pro{\bar{\theta}}{I} \label{ec:norm}\\
\pro{\theta_A,\theta_B}{I}   &= \pro{\theta_A}{\theta_B,I}\times\pro{\theta_B}{I}
%
\end{align}
\end{subequations}
%
\definition{$\pro{\theta}{I}$}{es nuestro grado de conocimiento sobre una hipótesis o premisa
$\theta$ condicionado sobre $I$. En el sentido más general $\theta$ puede representar un
vector, $\hip$, en el espacio de hipótesis, $\Hip$.}
%
En lenguaje cotidiano, la primera ecuación nos dice que dado nuestro grado de conocimiento
(probabilidad) de conjunto exhaustivo y mutuamente excluyente de posibilidades, podemos asignar
inmediatamente nuestro grado de desconocimiento (o ignorancia sobre el mismo conjunto). Ambas
cantidades sumadas forman una \emph{certeza}; la segunda ecuación nos dice que dado nuestro grado de
conocimiento de $\theta_B$ y nuestro grado de conocimiento en $\theta_A$ condicionado en
$\theta_B$, podemos admitir que sabemos nuestro grado de conocimiento de $\theta_A$ y
$\theta_B$.

%---------------------------------------------------------------------------------------------------

\section*{El teorema de Bayes}
%
\definition{$\pos{\hip}{\dat,I}$}{es la distribución de probabilidad posterior y representa
nuestro grado de conocimiento de la hipótesis a la luz de los datos, $\dat$.}
%
El teorema de Bayes se desprende de estas dos simples relaciones y sostiene que
%
\begin{equation}\label{ec:bayes}
%
\pos{\hip}{\dat,I} = \frac{\lik{\dat}{\hip,I}\times\pri{\hip}{I}}{\pro{\dat}{I}},
%
\end{equation}
%
donde
%
$$\pro{\dat}{I} = \int_{\Hip}\pos{\hip}{\dat,I}d\hip.$$
%
\definition{$\pro{\dat}{I}$}{se conoce como la evidencia y es la distribución posterior
marginalizada sobre el espacio de parámetros.}
%
El poder del teorema de Bayes en sí yace en nuestra capacidad para asignar las distribuciones de
probabilidad en el lado derecho de la igualdad, $\lik{\dat}{\hip,I}$ y $\pri{\hip}{I}$, en la
Ec.~\eqref{ec:bayes}. Esto es usualmente cierto en el caso de la última distribución, pues esta
refleja simplemente \emph{nuestro grado de conocimiento (ignorancia y/o prejuicios) sobre problema
previo a la obtención de los datos}, la primera distribución de probabilidades por otra parte,
requiere un poco más de elaboración y conocimiento (probablemente también prejuicios) sobre los
datos, pues representa la \emph{plausibilidad de la hipótesis asumida a la luz de los datos}.

%---------------------------------------------------------------------------------------------------

\section*{Inferencia bayesiana}
%
La inferencia bayesiana es la técnica que consiste en calcular la distribución de probabilidad
posterior, $\pos{\hip}{\dat,I}$ para hacer algún avance en nuestro grado de conocimiento sobre
$\hip$. El primer paso es construir la función que describe la probabilidad de haber hecho la
observación suponiendo la hipótesis como correcta y la distribución prior que describe nuestro grado
de conocimiento de la hipótesis antes de haber hecho la observación (véase el paréntesis A). Ambas
distribuciones de probabilidad requieren de nuestro conocimiento del problema, pero en particular,
la distribución \emph{likelihood} solo se puede concebir correctamente si conocemos también los
datos (e.g., los errores son Gaussianos, las observaciones son naturalmente poco probables, el
número de posibilidades es bajo, etc.).

La elección de la distribución de probabilidad prior, por otra parte, representa un problema más
complejo y que muchas veces es desestimado. Existen básicamente dos clases de distribuciones prior:
las subjetivas que permiten que el científico introduzca sus prejuicios y las objetivas
caracterizados generalmente por una distribución plana con la que solo se proporciona un rango de
plausibilidad para la hipótesis en cuestión, permitiendo así que los datos influyan más sobre la
distribución posterior. Por supuesto, como veremos a continuación, el efecto de una distribución
prior subjetiva se disipará en la medida en que el volumen de datos sea más grande, i.e., el efecto
de los datos dominara sobre nuestro grado de conocimiento de los mismos a través de la hipótesis
planteada.

Es importante notar que no todo prior plano es objetivo, si por ejemplo, el rango de plausibilidad
se restringe a uno más corto que el rango que incluye todas las posibilidades, entonces la
distribución prior es subjetiva. De igual manera, una distribución prior no plana no necesariamente
es subjetiva, pues una distribución prior pudo ser la distribución posterior de una inferencia
anterior a la obtención de los datos actuales o incluso, pudo ser la posterior de un problema
distinto. En estos casos decimos que la distribución prior está introduciendo dominio de
conocimiento sobre el modelo.

%---------------------------------------------------------------------------------------------------

\paragraph{\textsc{\color{Blue}Ejemplo: regresión lineal.}} Ahora vamos a hacer el ejercicio más
vulgar y silvestre y, aún así uno de los más útiles si se hace bien: ajustar una línea recta a un
conjunto de datos. Supongamos que dicho conjunto de datos está descrito por un arreglo independiente
(i.\,e., dado) $\set{x_i}$ y un arreglo de medidas $\set{y_i}$ cuyas desviaciones estándar están
dadas también $\set{\sigma_i}$.
% el conjunto de datos pueden ser 10 puntos en el rango x=(1,10) y y=2*x^0.9+0.5
% el error en y viene de una distribución que no es Gaussiana.
% mostrar el conjunto de datos en un gráfico
La intuición nos dice que por allí podría pasar una línea recta. La forma más objetiva de plantear
el problema, desde el punto de vista probabilístico, consiste en construir un modelo generativo
\citep{Hogg2010}: una descripción estadística que permite reproducir los datos observados. Esto es
esencialmente, construir la distribución de verosimilitud. Bajo la suposición de que los errores en
$\set{y_i}$ son Gaussianos y no correlados, la verosimilitud es simplemente:
%
$$\lik{\set{y_i}}{\set{x_i},\set{\sigma_i},m,b} = \prod_{i=1}^N\frac{1}{\sqrt{2\pi\sigma_i^2}}\exp{\left[-\frac{\left(y_i-m\,x_i+b\right)^2}{2\sigma_i^2}\right]}.$$
%
En este punto el problema ya se ha convertido en un tecnicismo: ¿cómo encontrar los parámetros $m$
y $b$, tal que la verosimilitud es máxima? La respuesta fácil (y muchas veces conveniente) a esa
pregunta es simplemente minimizar la función de mérito $\chi^2$.
% mostrar el ajuste lineal y los parámetros resultantes
Uno podría, sin razón lógica aparente, simplemente ir directo a la inferencia Bayesiana y encontrar
la distribución posterior $\pos{m,b}{\set{y_i},\set{x_i},\set{\sigma_i}}$ usando distribuciones
\emph{prior} objetivas, por ejemplo.
% mostrar la evolución de las posterior en m y b.
Sin embargo, ¿qué pasa si realmente no estamos seguros de que
$\lik{\set{y_i}}{\set{x_i},\set{\sigma_i},m,b}$ realmente es un modelo generativo de todo el
espacio de $\set{y_i}$? Y nos hacemos esta pregunta independientemente del la forma en que estén
relacionados los parámetros de interés.
% mostrar la distribución real de parámetros m y b.
Una forma de resolver este tipo de problemas es usando inferencia Bayesiana.

Bajo el esquema Bayesiano, las distribuciones \emph{prior} no solo se usan para introducir
conocimiento previo a la obtención de los datos, sino que al mismo tiempo se introducen
incertidumbres (de acuerdo con la definición de probabilidad). Podemos usar esto a nuestro favor.
% probar ajustar el mismo modelo usando un prior parecido a la verdadera distribución de errores.

%---------------------------------------------------------------------------------------------------

\section*{El efecto del prior}
%
En esta sección veremos como la elección de la distribución prior puede afectar significativamente
nuestra inferencia y como depende dicho efecto en la cantidad de los datos y en la forma del prior.

Para cuantificar el efecto del prior, estudiaré la dependencia de la diferencia entre las
distribuciones posteriores obtenidas usando un prior objetivo y uno subjetivo, como función del
número de datos observados.

\paragraph{\textit{\color{teal}Dependencia con la cantidad de los datos.}} Supongamos que la
distribución \emph{likelihood} dado un conjunto de datos $\dat$ es
%
$$\lik{\dat}{\hip,I} = \prod_{i=1}^N\lik[i]{\mathcal{D}_i}{\hip,I},$$
%
entonces la log-posterior es simplemente
%
$$\log{\pos{\hip}{\dat,I}} = \sum_{i=1}^N\log{\lik[i]{\mathcal{D}_i}{\hip,I}} + \log{\pri{\hip}{I}} + K,$$
%
donde $K$ es una constante. Es claro entonces que la distribución posterior escala con $N$, el
número de datos. Esto tiene sentido intuitivamente hablando, porque es de esperarse que en el límite
de $N\to\infty$ los datos sean lo suficientemente informativos a través de la verosimilitud como
para anular la contribución de la distribución previa.
% Mostrar ejemplo de la actualización de la distribución posterior al añadir observaciones.

\paragraph{\textit{\color{teal}Dependencia calidad de los datos.}} Si los datos son de baja calidad,
e.\,g. tienen incertidumbres típicas muy altas, la contribución de la verosimilitud de nuevo se ve
comprometida, más aún en caso de una distribución previa sea no objetiva.
% Mostrar ejemplo de la distribución posterior usando datos con distinto valor de la S/N.

%---------------------------------------------------------------------------------------------------

\section*{¿Cómo elegir \pri{\hip}{I}?}
%
Por supuesto la elección de la distribución de probabilidad previa depende del problema en
particular que se quiera resolver, pero sobre todo del conocimiento que se posea en el espacio de la
hipótesis. Por ejemplo, si nos sentimos muy seguros sobre nuestros prejuicios podríamos sentirnos
tentados a asumir una distribución previa informativa,
% Mostrar un ejemplo de una distribución informativa
mientras que si por el contrario tenemos poco conocimiento sobre el problema lo mejor sería asumir
una distribución previa objetiva.
% Mostrar un ejemplo de una distribución objetiva
Pero ¿cómo representar lo que sabemos o ignoramos en forma de distribución de probabilidad de la
forma más objetiva posible? Bueno, en el caso de una distribución previa objetiva es obvio: usamos
el ``principio de razón insuficiente'', el cual supone que dado un conjunto exhaustivo y mutuamente
excluyente de posibilidades, si no hay razones para pensar que una posibilidad es más probable que
otra, entonces lo más justo es asignar la misma probabilidad a todas. Pero ¿qué pasa si en efecto
poseemos razones suficientes para pensar que la distribución no es plana?

\paragraph{\textit{\color{teal}Información y entropía.}} En la ciencia la observación de un evento
casi siempre proporciona información que eventualmente nos permite actualizar nuestro grado de
conocimiento sobre ese evento. Cuando uno diseña un experimiento, uno de los pasos es plantear una
hipótesis que refleje nuestro conocimiento previo a la observación del evento. Estadísticamente
hablando esta hipótesis es descrita por la probabilidad de ocurrencia de dicho evento. Pero ¿cuál es
la relación entre la información adquirida tras la observación y la probabilidad previa de
ocurrencia?

Supongamos que la probabilidad previa de un evento $\theta$ es $\pro{\theta}{I}$, ¿cuánta
información obtenemos si observamos que el evento ocurre? Bueno, Shannon (quien diseñó la teoría de
información) planteó un conjunto de reglas simples que debe cumplir la función de información
$I[\pro{\theta}{I}]$ suponiendo que esta solo dependa de la probabilidad previa.
%
\begin{description}
%
\item[Dominio:] La información depende de $\pro{\theta}{I}$ únicamente, de manera que su dominio es
$[0,1]$.
% Ilustrar el dominio.
\item[Rango:] La información debe ser positiva ($I\geq0$). Si se tiene la certeza de que un evento
ocurre, entonces la información obtenida tras su observación es $I(1)=0$. Mientras menor es la
probabilidad previa de observar un evento, la información obtenida tras su observación es mayor.
% Ilustrar el rango.
\item[Monotonía y continuidad:] La información debe ser monótona y continua, esto es, pequeños
cambios en $\pro{\theta}{I}$ deben reflejar pequeños cambios en $I$ y \emph{vice versa}.
% Ilustrar varias funciones que cumplen con estas premisas.
\item[Aditividad:] Si dos eventos independientes tienen probabilidad $\pro{\theta_A}{I}$ y
$\pro{\theta_B}{I}$ de ocurrir, entonces su probabilidad conjunta es el producto de sus
probabilidades individuales y la información provista por la observación de ambos eventos es
$I_{\theta_A,\theta_B} = I[\pro{\theta_A}{I}] + I[\pro{\theta_B}{I}]$.
% Resaltar la función que cumple con todas las premisas.
\end{description}

La última regla restringe considerablemente las posibilidades para $I$, de manera que la forma de
representar dicha información (demostrada por Shannon) es simplemente:
%
$$I[\pro{\theta}{I}] \equiv -\log{\pro{\theta}{I}}.$$
%
Ahora supongamos que tenemos un conjunto exahustivo de eventos mutuamente excluyentes cuyas
probabilidades previas forman una distribución de probabilidad $\pro{\theta}{I}$, la información
esperada debida a la observación de alguno de tales eventos es:
%
$$H[\pro{\theta}{I}] \equiv -\int \pro{\theta}{I}\,\log{\pro{\theta}{I}}\,\text{d}\theta.$$
% Motrar un ejemplo con una Gaussiana y con una distribución plana
Esta cantidad es la denominada entropía de una distribución de probabilidad y es una medida de la
impredictibilidad de los eventos que representa dicha distribución. Pero ¿qué propiedades posee la
entropía? y, en particular ¿la entropía posee un máximo? (ya veremos por qué nos interesa
precisamente el máximo) Uno puede demostrar fácilmente que la entropía tiene un máximo usando la
desigualdad de Gibbs:
%
$$\int \pro{\theta}{I}\,\log{\left[\frac{1/(\theta_\text{max}-\theta_\text{min})}{\pro{\theta}{I}}\right]}\,\text{d}\theta\leq0,$$
% Motrar curvas y=log(x) y y=x - 1 y el punto donde son iguales
con la igualdad si y solo si $\pro{\theta}{I}=1/(\theta_\text{max}-\theta_\text{min})$.

\paragraph{\textit{\color{teal}Método de Máxima Entropía.}} En el apartado anterior vimos que dada
la probabilidad previa de ocurrencia de un evento se puede cuantificar la información que se obtiene
tras su observación simplemente usando la probabilidad asignada. Además, la entropía es la
información que se espera obtener dada una distribución de probabilidad. En este sentido, parece
coherente que una forma justa de asignar la distribución previa, $\pri{\theta}{I}$ consista en
maximizar la entropía, esto es, maximizar la información que se obtiene una vez se ha hecho una
observación.

Supongamos que tenemos información contrastable sobre un conjunto exhaustivo de eventos mutuamente
excluyentes $\hip$ y queremos asignar una distribución de probabilidad que cumpla con esa condición
y que a la vez garantice que posterior a la observación tendremos la máxima información posible. El
método de máxima entropía consiste en optimizar la función:
%
$$L(Pr;\theta,I) = \int\pro{\theta}{I}\,\log{\left[1/\pro{\theta}{I}\right]}\,\text{d}\theta - \lambda_0\left[\int\pro{\theta}{I}\,\text{d}\theta-1\right] - \sum_{i=1}^N \lambda_i\,g_i(Pr;\theta,I),$$
%
donde el primer término es la entropía, el segundo es la condición de normalización y es una
restricción necesaria y la sumatoria representa las restricciones impuestas por la información
contrastable encapsulada en las funciones $g_i(Pr;\theta,I)$. Los factores $\lambda_i$ son los
multiplicadores de Lagrange $\lambda_i$.

\paragraph{\textsc{\color{Blue}Ejemplo: función inicial de masa.}} En este ejemplo vamos a
reconstruir la FIM usando el método de máxima entropía. Como información verificable

%---------------------------------------------------------------------------------------------------

\section*{¿Qué tan bueno es el modelo?}

La distribución posterior proporciona estimaciones de los parámetros y de sus incertidumbres,
incluyendo la propagación de las incertidumbres observacionales, de manera consistente y confiable,
siempre y cuando la distribución previa comprenda todas las posibilidades en el espacio de
hipótesis. Sin embargo, por sí misma la distribución posterior no contiene información sobre si el
modelo adoptado es bueno o malo.

Una forma de cuantificar la plausibilidad del modelo adoptado consiste en muestrear la distribución
posterior predictiva, esto es:
%
\begin{equation}
\pro{\dat[j]^\text{pre}}{\dat,I} = \int\pro{\dat[j]^\text{pre}}{\hip,I}\times\pos{\hip}{\dat,I}\text{d}\hip,
\end{equation}
%
donde el primer factor dentro de la integral es la probabilidad de haber observado
$\dat[j]^\text{pre}$ dada la hipótesis. La premisa es que si el modelo es descriptivo de las
observaciones en su completitud, entonces una muestra tomada de la distribución
$\pro{\dat[j]^\text{pre}}{\hip,I}$ debería ser indistinguible de las observaciones reales. Para
verificar que esto es así, se ejecutan los siguientes pasos:
%
\begin{description}
%
\item[Conjunto $\set{\hip_j}^{\rm pre}$.] Necesitamos un conjunto actualizado (posterior) de
parámetros en el espacio de hipótesis. Así que muestreamos la distribución posterior.
%
\item[Conjunto $\dat^{\rm pre}$.] Usando el conjunto $\hip^\text{pre}$ del paso anterior, se
construye un conjunto de observaciones predichas, muestreando la distribución
$\mathcal{P}\left(\mu\,\middle|\,\hip^\text{pre},I\right)$. Por supuesto
$\pro{\dat[j]^\text{pre}}{\hip,I}$ y $\mathcal{P}$ deben ser consistentes: si la probabilidad del
evento $\mathcal{D}_i$ se distribuye alrededor de un promedio $\mu_i$ con desviación estándar
$\sigma_i$, entonces la elección sensible para $\mathcal{P}$ sería una distribución Gaussiana y la
verosimilitud sería muy parecida a la del problema de ajustar una línea recta que vimos antes.
%
\end{description}

%---------------------------------------------------------------------------------------------------

\section*{Resumen}

\end{document}
