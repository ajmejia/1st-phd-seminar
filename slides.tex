\documentclass[a4paper,twoside]{article}

\usepackage{amsmath}
\usepackage[spanish]{babel}
\usepackage[T1]{fontenc}
\usepackage[utf8]{inputenc}
\usepackage{pxfonts}

\usepackage{marginnote}
\usepackage[top=1.5cm, bottom=1.5cm, outer=5cm, inner=2cm, heightrounded, marginparwidth=3cm, marginparsep=1.5cm]{geometry}

\title{Galaxias Tempranas}
\author{Alfredo Mejía-Narváez}
\date{\today}

\begin{document}

\maketitle

\begin{abstract}
Las galaxias tempranas son menos simples de lo que las imágenes fotométricas dejan ver. Con una apariencia sin estructuras a simple vista, poco gas frío y polvo y una casi despreciable formación estelar, estas galaxias están dominadas por una población estelar vieja y rica en metales. 
\end{abstract}

\section{Morfología}

Las galaxias tempranas están entre las más promientes en términos de la luminosidad integrada y de la masa estelar. Estructuralmente no parecen muy complejas, a diferencia de sus semejantes más masivas, las galaxias espirales o incluso las lenticulares (S0) cuya luminosidad puede estar dominada por un bulbo estelar o por un disco estelar, ambos siempre presente en estos tipos de galaxias. Las galaxias tempranas, como su nombre sugiere, poseen una morfología a simple vista esferoidal, generalmente con geometría triaxial. En la Clasificación Morfológica de Hubble, se les denomina galaxias tempranas por razones históricas.

Se ha observado que existe una fuerte correlación entre el ambiente y la morfología en que este tipo de galaxias se encuentran, en términos de la densidad de número de galaxias y la distribución de colores en los cúmulos de galaxias; las galaxias tempranas tienden a concentrarse en las regiones de mayor densidad hacia el centro de los cúmulos de galaxias, mientras que sus homólogas espirales, tienden a encontrarse con mayor probabilidad en las zonas de menor densidad, hacia las afueras de los cúmulos.

La morfología también está estrechamente relacionada con el color integrado, como sugerí en el párrafo anterior; la forma del espectro integrado (en el óptico) está prácticamente determinada por su contenido estelar únicamente, con líneas de absorción típicas de poblaciones estelares viejas y ricas en metales. Aún así se ha encontrado que en la región espectral de altas energías (rayos X), existe emisión del gas ionizado. \marginnote{¿Qué fuentes producen la ionización del gas?}[0cm]

\section{Estructura, cinemática y dinámica}

Las galaxias elípticas parecen simples en apariencia, pero estudios fotométricos de su estructura y campo de velocidades han revelado que existen distintas subclases de galaxias tempranas. Una observación relativamente directa de hacer es que las galaxias tempranas se pueden dividir en rotadores regulares y en rotadores no regulares. Más tarde los sondeos usando IFUs permitieron hacer una distinción más fina entre las galaxias rotadoras no regulares, estableciendo finalmente las siguientes subclases de rotadoras no regulares: galaxias no rotadoras, galaxias con velocidad compleja, galaxias con núcleos cinemáticamente distintos/desacoplados (KDCs en inglés), y galaxias con dos discos contra-rotadores.

Aunque la distinción estructural de galaxias tempranas entre rotadoras regulares y no regulares puede conllevar a varias soluciones posibles usando modelos puramente dinámicos, dado que en principio no existe razón física para que la distribución de estrellas no sea triaxial, el hecho de que exista poca desviación entre el semineje mayor fotométrico y el correspondiente de la velicidad es un claro indicador de que la estructura de las galaxias tempranas rotadoras regulares debe poseer simetría alrededor de un eje. Sin embargo, la degeneración que impide reconstruir la estructura de las galaxias tempranas rotadoras no regulares permanece, en principio. Aun así, la cinemática observada en estos sistemas ha permitido restringir la relación entre el semieje menor y el semieje mayor $c/a>0.65$. Esto es además confirmado con la medición de la elipticidad en rotadores no regulares, siendo esta $\epsilon<0.4$, lo que demuestra que este tipo de sistemas es merginalmente triaxial.

El ajuste de modelos dinámicos ha permitido una comprensión más detallada (dentro de los límites impuestos por las distintas degeneraciones) de la estructura de las galaxias tempranas en función de su cinemática. Las observaciones han mostrado que las galaxias con simetría cilíndrica (axisimétricas) poseen las mayores anisotropías en su distribución 3D de velocidades. De la muestra de galaxias estudiadas, dos galaxias (NGC 4550 y NGC 4473) mostraron un comportamiento distintivo, con dos picos simétricos en la distribución de la dispersión de velocidades en la dirección del semieje mayor, lo que sugería la presencia de dos discos en contra rotación. Esta interpretación fue posteriormente confirmada usando modelos dinámicos, donde se ajustaron dos poblaciones estelares claramente rotando en sentidos opuestos. Galaxias de este tipo (apodadas $2\sigma$) constituyen el $4$ por ciento de la muestra estudiada. En el caso de las rotadoras no regulares más circulares (en apariencia) las anisotropías se distribuyen alrededor de cero, indicando que estas son cercanamente esféricas.

Un tipo de galaxias rotadoras no regulares son las KDCs. Modelos dinámicos ajustados a la cinemática de estos objetos han demostrado que estas son un tipo especial de galaxias con dos poblaciones estelares en contra rotación, cuyas velocidades medias se cancelan unas a otras en una zona extendida alrededor del núcleo. Sin embargo, este aparente desacoplamiento de la región central en este tipo de galaxias no es debido a diferencias importantes en la distribución de velocidades. \marginnote{Revisar los mecanismos de formación de este tipo de galaxias.}[0cm]

El hecho de que existan diferencias tan relevantes en la estructura de galaxias tempranas es evidencia de que estas probablemente se han formado mediante canales distintos, pues la escala temporal dinámica es mayor que la edad del universo y, por lo tanto no existe posibilidad de que estos sistemas estelares representen distintas etapas evolutivas de una misma clase de galaxias. \marginnote{Revisar el gráfico de la dicotomía.}[0cm]

\section{Poblaciones estelares}

Los primeros estudios sobre el contenido estelar en galaxias tempranas se hizo estimando la edad y la metalicidad estelares usando los índices de Lick. Hoy en día se sabe que este tipo de galaxias están dominados por una población estelar vieja ($\sim10^{9}$ años) sin evidencia de eventos de formación estelar reciente. La población estelar es también rica en metales (incluso varias veces por encima de Z${}_\odot$), que abarcan un amplio rango de masas estelares ($10^9$---$10^{12}$ M${}_\odot$). \marginnote{Revisar ese rango de masas.}[-2cm] Adicionalmente, en las galaxias más masivas se ha encontrado que existe una sobreabundancia de elementos $\alpha$, e.g. Mg, en comparación con la abundancia de Fe, lo que sugiere un \emph{$\alpha$-enhancement} de la población estelar debido a supernovas del tipo II. Más adelante veremos cómo se intepreta este interesante resultado en términos de la evolución de las poblaciones estelares en las galaxias tempranas. \marginnote{Elaborar sobre los observables que las hacen distintas de otras clases morfológicas de galaxias.}[-2cm]

Los estudios del contenido estelar en galaxias en general han revelado relaciones de escala existentes entre cantidades observables y/o derivadas (usando la técnica del ajuste espectral, por ejemplo) con profundas implicaciones físicas sobre la formación y la evolución de las galaxias, algunas aún sin ser totalmente comprendidas y en las que las teorías de formación y evolución de estructuras en el universo estándar fallan. Ejemplos de estas relaciones de escala, en el particular caso de las galaxias tempranas son, la relación edad-metalicidad, en la cual las galaxias muestran una anticorrelación; las galaxias dominadas por poblaciones más viejas son también menos ricas en metales, mientras que las dominadas por poblaciones más jóvenes ($>5$ Gaños) muestran un enriquecimiento químico más temprano. No parece haber una dependencia fuerte con la masa estelar en esta tendencia. \marginnote{¿Esto se debe a la degeneración edad-metalicidad?}[-2.5cm]. Otra correlación fuerte en el caso de las galaxias tempranas se ha visto en el espacio $()$ donde estas se distribuyen en un plano bien definido llamado Plano Fundamental. \marginnote{Elaborar más sobre el Plano Fundamental.}[-1cm]

\section{Galaxias homólogas y progenitoras a $z>0$}

Hasta $z\sim1$ las poblaciones estelares en galaxias tempranas lucen esencialmente iguales, lo que ha sido interpretado como un signo de evolución pasiva de las poblaciones estelares, sin mayores eventos de formación estelar. Más interesante es el hecho de que la distribución de colores en la secuencia roja de galaxias en el diagrama color-magnitud muestra el mismo comportamiento a diferentes valores de $z$, lo que es un claro indicador de que la fuerte correlación entre la edad y la metalicidad no es la responsable de la secuencia roja, pues es sobservaría un empinamiento de esta con $z$ creciente; La distribución de colores se haría más azul en función del corrimiento al rojo. Adicionalmente estudios de la luminosidad característica de la función de luminosidad, como función del corrimiento al rojo también muestran una evolución pasiva de las poblaciones estelares en galaxias tempranas localizadas en cúmulos desde $z\sim1$; las luminosidad característica muestra un incremento con $z$ creciente. Una implicación de este comportamiento es que la mayoría de las galaxias (al menos en los centros de sus respectivos cúmulos) se habrían ensamblado cuando el universo tenía la mitad de su edad actual.

Estudios del Plano Fundamental en función del \emph{redshift} muestra un comportamiento de nuevo en acuerdo con una evolución pasiva de las poblaciones en galaxias tempranas; el Plano Fundamental se desplaza casi paralelamente a sí mismo de manera que con $z$ creciente, la separación aumenta. El hecho de que la rotación del Plano Fundamental con el incremento en \emph{redshift} sea mínimo es evidencia de que la relación edad-$\sigma$ derivada de los estudios de los índices de Lick no es consistente con el comportamiento observado. Más aún, el hecho de que la dispersión perpendicular al plano permanezca prácticamente invariante con el incremento de $z$ es evidencia de que la anticorrelación edad-metalicidad encontrada usando los mismos índices espectrales pudiera no interpretarse como una conexión causal entre ambas propiedades. De lo contrario el plano rápidamente desaparecería a $z>0$. Por otra parte, ya se han observado diferencias claras entre las galaxias tempranas de campo y las ligadas a un cúmulo, e.g., las galaxias de campo son sistemáticamente menos masivas y están dominadas por una población en promedio más joven (por $\sim1$ Gaño). Sin embargo, las galaxias más masivas en el campo parecieran comportarse más similarmente a sus análogas en los cúmulos. En cualquier caso, se ha encontrado que tanto galaxias en el campo como en cúmulos deben haber ensamblado la mayor parte de su masa a $z\sim2$ y que el seso de su formación estelar procede desde las más masivas hasta las menos masivas, como es de esperarse por el efecto de \emph{downsizing}.

En resumen, el estudio de galaxias tempranas en cúmulos hasta $z\sim1$ muestra claramente que estos objetos han evolucionado pasivamente (al menos) desde $z\sim2$---$3$, y que la debieron haber alcanzado su masa `observada' a $z\sim1$. Aunado al efecto de \emph{downsizing}, en el que las galaxias más masivas dejan de formar estrellas antes que las menos masivas, las observaciones muestran que las galaxias más masivas en cúmulos se formaron en escalas temporales más cortas que las galaxias menos masivas, las cuales muestran (de acuerdo con sus índices espectrales) una población estelar de edad intermedia ($>5$ Gaños) \marginnote{Revisar esa edad.}[0cm]. Es importante destacar que estos resultados están afectados por el sesgo de morfología, ya que muchas de las galaxias tempranas en el universo local podrían tener su homóloga a $z\sim1$ morfológica y fotométricamente diferente.

A \emph{redshift} más grandes, la detección de progenitoras de galaxias tempranas ha sido más complicada, en parte por las limitaciones tecnológicas y en parte por limitaciones introducidas por el sesgo de morfología. Sin embargo, observaciones de campo profundo disponibles en la actualidad han aliviado la primera limitación. Por otra parte, para romper con la limitación del sesgo morfológico \marginnote{Revisar que esto sea lo mismo que el sesgo del progenitor.}[0cm] hacer estimaciones de la densidad del número de galaxias tempranas como función del corrimiento al rojo; lo que se espera es que a partir de cierto valor de $z$, este comience a decaer, pues las galaxias progenitoras de elípticas se clasificarían como galaxias con formación estelar y aparecerían en el plano color-magnitud por debajo de la secuencia roja. En este sentido, los primeros resultados mostraron poca evolución de la densidad numérica de galaxias tempranas hasta $z\sim1$. Las observaciones de campo profundo del Hubble Space Telescope (HST), permitieron la detección de objetos extremadamente rojos (EROs, en inglés) con características fotométricas compatibles con lo esperado para progenitores de galaxias tempranas. Se encontró que aproximadamente un $50$ por ciento de estos objetos detectados a $z\sim1$ son efectivamente pasivos y probables progenitores de las galaxias tempranas que vemos en el universo local y que debieron formarse a $z>3$; mientras que la otra mitad se corresponde con objetos con altas tasas de formación estelar, enrojecidos por polvo (similares a las ULIRGs). Además, se encontró luego que estos objetos (EROs) eran más frecuentes de lo que se había observado en los campos del HST y curiosamente su densidad numérica es consistente con una evolución lenta con el \emph{redshift} observada en galaxias tempranas.

El hecho de que las galaxias más masivas ya se han incorporado a la secuencia roja a $z\sim1$ merece especial atención. Este hecho pareciera sugerir que el efecto del \emph{downsizing} de formación estelar se extiende al ensamblaje de galaxias tempranas; las galaxias más masivas se ensamblan primero y luego le siguen las menos masivas. Esto pareciera entrar en conflicto con el escenario de formación jerárquica de las galaxias, donde las galaxias más masivas se forman por la fusión de galaxias menos masivas, típicamente ricas en gas y con formación estelar reciente. De acuerdo con esta teoría, el producto final de tal fusión debería contener estrellas tan jóvenes como las de sus progenitoras, por lo tanto con estrellas no más viejas que unos cuantos Gaños. Sin embargo, para probar de manera unívoca que el efecto del \emph{downsizing} se extiende a escalas cósmicas es necesario estudiar la función de luminosidad y de masa de galaxias con muestras más completas, pues aún existen discrepancias entre los resultados de distintos autores.

\section{Formación y evolución de poblaciones estelares}

Es de esperarse que conforme miremos más profundo en el cielo en busca de galaxias tempranas, la cantidad de detecciones de estas disminuya, ya sea haciendo una clasificación morfológica o una fotométrica, esto independientemente de las capacidades del instrumento de observación. Es decir, en algún punto tenemos que comenzar a observar más galaxias progenitoras de las elípticas y menos de estas últimas, presumiblemente con una morfología y fotometría distintos. Encontrar estas galaxias daría pistas valiosas sobre la formación y evolución de las galaxias tempranas.

En el caso de las galaxias S0, la solución al problema de las galaxias progenitorias parece sencillo. En el rango de $z<1$ es claro que el número de galaxias S0 disminuye, con \emph{redshift} creciente, a la vez que el número de galaxias espirales aumenta. Es obvio entonces que las galaxias tempranas eventualmente se conviertan en galaxias S0. En el caso de las galaxias esferoidales, como las elípticas esto no es tan sencillo, pues comprenden un amplio rango de luminosidades y, por lo tanto de masas y aunque tienen todas la misma simetría, tienen parámetros estructurales (e.g. el radio de media luz y el brillo superficial), cinemáticos y dinámicos (velocidad de rotación, distribución de velocidades y órbitas) que abarcan amplios rangos de valores. El hecho de que estos sistemas tengan un tiempo de relajación dinámica más prolongado que la edad del universo sugiere que estas diferencias no se deben a que estamos viendo el mismo tipo de sistema en distintas fases evolutivas (como en la secuencia principal de estrellas en el diagrama H-R), sino que estas tuvieron un origen distinto. Por lo tanto buscar los progenitores en un solo tipo fotométrico y/o morfológico pareciera ser una necedad.

Los primeros esfuerzos en este sentido se concentraron en las galaxias con formación estelar detectadas usando el método de Lyman Break ($\sim3$), las llamadas Galaxias Lyman Break (LBGs en inglés). Sin embargo, el número de LBGs no era compatible con lo que se esperaba detectar a un dado corrimiento al rojo. Eventualmente se observó que limitarse en este tipo de galaxias introducía un sesgo que impedía detectar las galaxias más masivas y con tasas de formación estelar más altas ($\sim200$ M${}_\odot/$año). \marginnote{Revisar por qué esto es así.}[0cm] En este sentido las ULIRGs parecían ser buenas candidatas para evolucionar en galaxias tempranas masivas ($>10^{11}$ M${}_\odot$). Otro rasgo que hacía a este tipo de galaxias mejores candidatas para ser progenitoras de galaxias tempranas era su \emph{clustering} que a $z>2$ es mayor que el de las LBGs, pues se sabe que las galaxias tempranas en el universo local son las más conglomeradas. A pesar de que las ULIRGs tienen una densidad numérica que es diez veces menor que la de EROs ($z\sim1$), estas galaxias se piensa que evolucionan en una escala temporal corta debido a su alta tasa de formación estelar.

El diagram $BzK$ se ha convertido en una herramienta invaluable en la detección de galaxias con altas tasas de formación estelar a $z>2$. Esto último se ha confirmado mediante la detección en el submilimétrico de objetos previamente clasificados como galaxias $BzK$ con formación estelar. De hecho el número de galaxias masivas ($>10^{11}$ M${}_\odot$) con formación estelar es muy similar al número de galaxias pasivas, ambos clasificados con dicho diagrama. Ambos cantidades sumadas se corresponden bien con la densidad numérica de galaxias tempranas en el universo local. Más interesante aún es que una muestra de galaxias distantes rojas (DRGs en inglés) parecen corresponderse con galaxias con altas tasas de formación estelar y enrojecidas por polvo interno. Dicha muestra pareciera extenderse en un amplio rango de corrimientos al rojo ($z=1$---$3.5$) con una distribución relativamente plana.

Finalmente otro mecanismo mediante el cual se pueden formar las galaxias tempranas que vemos hoy a $z=0$ es el llamado \emph{dry merger} en el cual dos galaxias tempranas se fucionan para dar lugar a otra más masiva. Sin embargo, se ha encontrado que la probabilidad de que este tipo de evento ocurra en el rango $z<1$, donde es más plausible encontrar este tipo de galaxias bien conglomeradas, es apreciablemente baja.

\section{Como encajan las piezas}

A finales de la década de los 70's se había observado que la formación estelar ocurría exclusivamente en discos (galaxias tardías), por lo tanto fusionar dos galaxias de discos para producir una galaxia elíptica parecía un escenario plausible para la formación de sistemas tempranos. Sin embargo hoy sabemos que la formación estelar exclusiva en los discos ocurre solo a $z<1$ y que $z=1$---$1.5$ la mayor parte de la formación estelar ocurre en galaxias \emph{starburst} como las ULIRGs. Sabemos también que $\sim50$ por ciento de las estrellas se formaron en la era de las galaxias con discos ($z<1$) y que la mayor parte de las galaxias tempranas ya se había ensamblado a $z\sim1$, tal vez antes. \marginnote{Esto parece referirse al efecto de \emph{downsizing} en la formación y ensamblaje de galaxias. La formación de galaxias masivas ocurre primero y luego la formación de galaxias menos masivas. La formación de estas últimas se prolonga hasta el presente.}[-6cm]

\section{Sondeos con IFU}
Los sondeos con IFU han revolucionado nuestra visión de las galaxias. En particular, en el caso de las galaxias tempranas los sondeos con IFU nos han permitido el estudio detallado del contenido estelar (y de gas) en las galaxias cercanas y de su cinemática interna. En consecuencia los modelos dinámicos, basados en las medidas cinemática mejoraron apreciablemente.

Un importante avance producido en parte por la disponibilidad de datos de sondeos con IFU fue la interpretación del Plano Fundamental. Desde finales de los 80's es conocido que las galaxias tempranas se distribuyen en el espacio $(L, \sigma, R_e)$ de manera que producen un plano con una dispersión estadística del $20$ por ciento en $R_e$. La interpretación de tal distribución fue que las galaxias tempranas obedecían la relación del equilibrio virial $M\propto\sigma^2R$. Sin embargo, se encontró que los exponentes de la relación $L\propto\sigma^aR_e^b$ no se correspondían con la ecuación de equilibrio virial y que había una desviación sistemática de esta interpretación. A esta desviación se le conoce como la inclinación del Plano Fundamental.

Entre las posibles causas de dicha inclinación se se contó la dependencia de la relación masa-luminosidad con la masa y con la dispersión de velocidades. A finales de los 80's ya era conocido este hecho y se estimaba que el incremento de $M/L$ como función de $\sigma$ podría explicar en parte la desviación del Plano Fundamental. También la dependencia del brillo superficial en la luminosidad, siendo los perfiles más concentrados en objetos más luminosos; para una masa fija, un perfil más empinado implica una dispersión de velocidades más grande en la región central de las galaxias, donde $\sigma$ es medido. Tal variación de $\sigma$ sería en principio suficiente para explicar la desviación del Plano Fundamental. Una tercera causa de tal desviación es la cantidad de materia oscura en la región central de las galaxias, que se sabe depende de la masa: a mayor masa, mayor es la fracción de materia oscura, lo que puede producir una variación en $M/L$ lo suficientemente importante como para desviar el Plano Fundamental como se ha medido. 

Usando modelos dinámicos ajustados a la cinemática de galaxias tempranas se encontró que de hecho existe una estrecha correlación entre $M/L$ y $\sigma$. Dicha correlación explica por completo la desviación del Plano Fundamental del equilibrio virial. Estos resultados fueron posteriormente reafirmados usando datos de sondeos con IFU y muestras más grandes de galaxias. En particular se mostró que cuando se usa la masa en lugar de la luminosidad en la ecuación del Plano Fundamental los exponentes ajustados se corresponden con el equilibrio virial (dentro de las incertidumbres). \marginnote{Revisar por qué el uso de la masa en la ecuación del Plano Fundamental recupera la correspondencia con el equilibrio virial.} Cabe preguntarse entonces sobre el origen físico de la desviación del Plano Fundamental, respecto de la forma en que es determinada la relación $M/L$. Por un lado se han encontrado discrepancias entre la relación masa luminosidad derivada mediante análisis de las poblaciones estelares en las galaxias y la derivada mediante modelos dinámicos. Esta última forma además premite determinaciones precisas de la masa total de las galaxias tempranas. Es claro entonces que las razones detrás de la aparente inconsistencia entre el Plano Fundamental y su interpretación como una consecuencia del equilibrio virial están relacionadas con la fracción de materia oscura y/o con variaciones de la función inicial de masa (IMF en inglés). De hecho, se ha encontrado que el efecto de la fracción de materia oscura dentro del radio de media luz (donde se miden los parámetros del Plano Fundamental) es despreciable para que los modelos puedan reporducir adecuadamente la fotometría y la cinemática observados. Esto deja entonces abierta la posibilidad de que variaciones en el contenido estelar, específicamente relacionados con la IMF sean los responsables por el comportamiento en el Plano Fundamental.

El camino hacia la interpretación de la desviación del Plano Fundamental del equilibrio virial cuando se usan los resultados inferidos de las poblaciones estelares luce aún incierto. Sin embargo, existen claros indicios de que la universalidad de la IMF (usualmente asumida) podría no cumplirse en galaxias tempranas, dando pie a que exista una dependencia entre la cinemática y la IMF, probablemente debida a diferencias en los canales de formación de las galaxias tempranas. Los sondeos con IFU como MaNGA lucen prometedores en este sentido, pues producirán datos espectrales en un amplio rango de longitudes de onda, incluyendo los rasgos más sensibles a las variaciones en la IMF.

\section{Limitaciones y Preguntas sin responder}

La detección de galaxias en el submilimétrico se ha hecho a cuenta gotas por la baja sensibilidad de los detectores.

Consistentencia entre la historia de formación estelar medida usando evidencia fósil y la medida a altos redshifts. Los resultados parecen corresponderse dentro de las incertidumbres, pero estas son demasiado altas aún como para sacar conclusiones.

El efecto de \emph{downsizing} pareciera no corresponderse con el paradigma de formación jerarquica de galaxias. Sin embargo, reproducir su efecto no ha sido un problema en el marco de dicha teoría, pues es intuitivo pensar que las zonas de mayor densidad producirán los eventos más importantes de formación estelar y se convertirán eventualmente en las galaxias que hoy vemos en la secuencia roja. Aún así, las simulaciones tuvieron que agregar un ingrediente extra para que las escalas a las cuales ocurren estos eventos se correspondieran con las observaciones, el llamado \emph{AGN feedback}, que suprime las corrientes de enfriamiento que eventualmente producen una formación estelar extendida, incluso hasta $z\sim0$. Aún hoy tal artificio físico parece incierto, sin embargo sabemos que las galaxias y su agujero negro central evolucionan a la par.

El origen físico de la desviación del Plano Fundamental de la predicción del equilibrio virial es aún incierto. Existen claras tensiones entre la relación masa-luminosidad que se calculan usando modelos de síntesis de poblaciones y modelos dinámicos dependencia clara en la dispersión de velocidades. Todo pareciera apuntar a (o al menos se puede explicar con) variaciones de la IMF.

Un problema aún más importante es que de acuerdo con la física de la materia oscura, los halos una vez formados tienden a fusionarse por inestabilidades gravitacionales, produciendo halos cada vez más masivos. Es de esperarse entonces que a medida que observemos más lejos (altos corrimientos al rojo) progresivamente dejemos de ver galaxias masivas. Aún así esto no ha ocurrido y si acaso, existe evidencia de que ocurre lo contrario.

\section{Resumen y perspectivas}

Las galaxias tempranas son mucho más complejas de lo que las observaciones fotométricas dejan ver. Incluso haciendo un análisis de las isofotas de una muestra pequeña de este tipo de galaxias deja ver que existen claras diferencias entructurales dentro de esta clasificación: las galaxias tempranas no son homólogas. Estudios cinemáticos, más aún usando las bondades de los sondeos con IFU, han permitido romper, al menos en parte, las degeneraciones en la clasificación morfológica convencional (la clasificación de Hubble) que impedían distinguir claramente los tipos de galaxias tempranas. Hoy día hablamos de galaxias sostenidas por rotación, con distribuciones orbitales en forma de disco, y galaxias soportadas por dispersión de velocidades, con órbitas aparentemente aleatorias distribuidas en una geometría esferoidal, típicamente triaxial. Esta dicotomía en la distribución de las galaxias tempranas sugiere mecanismos de formación distintos, pues el tiempo de relajación requerido para que un sistema evolucione de un estado a otro es mucho mayor que la edad del universo. Los sondeos con IFU también han permitido un refinamiento de los complicados modelos dinámicos que supuestos sobre estos sistemas, pues los exquisitos mapas de velocidades estelares producidos por estos sondeos sirven como entrada para el ajuste de los perfiles de masa, incluso en galaxias elípticas, cuyos modelos dinámicos son naturalmente más complejos.

A pesar de la no homología de las galaxias tempranas, se ha encontrado que estas galaxias comparten muchas propiedades físicas que pueden resumirse en su distribución en el espacio $(L, R_e, \sigma_e)$; el llamado Plano Fundamental y en sus respectivas proyecciones. La interpretación de dicho plano se ha hecho invocando el equilibrio virial. Sin embargo, ajustes de los exponentes de la relación entre los parámetros antes mencionados muestran desviaciones claras de esta interpretación: la llamada desviación del Plano Fundamental. Esto y la impresionantemente pequeña dispersión en $R_e$ a $L$ y $\sigma_e$ fijos son de las propiedades del Plano Fundamental sin una respuesta clara a la fecha.

Una forma de esquivar la interpretación de dicha desviación (y que tal vez a la larga ayude a explicarla) es usar la relacion masa-luminosidad para interpretar la distribución de galaxias en el espacio $(M, R_e, \sigma_e)$, naturalmente análoga al Plano Fundamental, pero con la gran ventaja de que permite (usando los exquisitos datos de los sondeos con IFU) usar un estimado de la masa total (bariónica y oscura) completamente independiente de los prejuicios de que plagan las estimaciones usando la relación masa-luminosidad de las poblaciones estelares (e.g. la universalidad de la IMF). Bajo estas circunstancias se ha encontrado que el Plano de Masa (MP en inglés) es completamente consistente con el equilibrio virial y que en sus proyecciones se manifiesta la dicotomía mencionada anteriormente. Las relaciones masa-velocidad y masa-tamaño, por ejemplo tienen la misma zona de exclusión de galaxias \marginnote{Revisar qué es exactamente esta zona.}[0cm] que se puede ajustar por dos leyes de potencias intersectadas en la masa $M_b\sim3\times10^{10}$ M${}_\odot$. Otra separación en masa importante ocurre a $M_\text{crit}\sim2\times10^{11}$ M${}_\odot$. Por debajo de esta masa crítica se encuentran las galaxias soportadas por rotación, con discos, incluyendo galaxias espirales; y por encima de la masa crítica se encuentran las galaxias soportadas por dispersión de velocidades, esferoidales y núcleos en su perfil de brillo superficial estelar. \marginnote{Acá núcleo se refiere al decrecimiento de la pendiente en el perfil del brillo superficial hacia el centro de estas galaxias.}[0cm]

La interpretación de estos resultados sugiere que las galaxias tempranas soportadas por rotación forman una secuencia paralela con las galaxias espirales tardías (dominadas por disco) y que en incremento de la dispersión de velocidades, $M/L$, edad, metalicidad, \emph{$\alpha$-enhancement} y fracción bulbo-disco; la morfología de las galaxias se transforma de dominadas por disco en dominadas por esferoides, todo esto aproximadamente en el rango de masas $2\times10^{9}$---$2\times10^{11}$ M${}_\odot$. En ese mismo rango solo las galaxias tempranas soportadas por rotación aparecen, mientras que las galaxias dominadas por dispersión de velocidades aparecen con $M>2\times10^{11}$ M${}_\odot$.

\end{document}
