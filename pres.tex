\documentclass[xcolor=dvipsnames,4pt,hyperref={colorlinks,citecolor=black,linkcolor=black,urlcolor=black}]{beamer}

\usepackage{amsmath}
\usepackage{eulervm}

\setbeamertemplate{navigation symbols}{}
\usefonttheme{professionalfonts}
\useinnertheme{circles}
\usepackage[spanish]{babel}
\usepackage[T1]{fontenc}
\usepackage[utf8]{inputenc}
\usepackage[orientation=portrait,size=custom,width=32,height=18]{beamerposter}

\usepackage{multirow}
\usepackage{graphicx}
\usepackage{tikz}

\usecolortheme[named=Black]{structure}

\usetikzlibrary{arrows,shapes,positioning,backgrounds}

\PassOptionsToPackage{%
%backend=biber, % Instead of bibtex
backend=bibtex8,bibencoding=ascii,%
language=auto,%
%style=numeric-comp,%
style=authoryear-comp, % Author 1999, 2010
bibstyle=authoryear,dashed=false, % dashed: substitute rep. author with ---
sorting=ynt, % year, name, title
maxbibnames=3, % default: 3, et al.
backref=false,%
natbib=true, % natbib compatibility mode (\citep and \citet still work)
url=false, %
doi=false, %
eprint=false %
}{biblatex}
\usepackage{biblatex}

\addbibresource{library.bib}

\hypersetup{
% Uncomment the line below to remove all links (to references, figures, tables, etc), useful for b/w printouts
%draft,
colorlinks=true, linktocpage=true, pdfstartpage=3, pdfstartview=FitV,
% Uncomment the line below if you want to have black links (e.g. for printing black and white)
%colorlinks=false, linktocpage=false, pdfborder={0 0 0}, pdfstartpage=3, pdfstartview=FitV,
breaklinks=true, pdfpagemode=UseNone, pageanchor=true, pdfpagemode=UseOutlines,%
plainpages=false, bookmarksnumbered, bookmarksopen=true, bookmarksopenlevel=1,%
hypertexnames=true, pdfhighlight=/O,%nesting=true,%frenchlinks,%
urlcolor=webbrown, linkcolor=RoyalBlue, citecolor=Maroon, %pagecolor=RoyalBlue,%
%urlcolor=Black, linkcolor=Black, citecolor=Black, %pagecolor=Black,%
}

% My default font
%\usepackage{newcent}

% Computer Modern Bright font
%\usepackage{cmbright}

% Iwona light
\usepackage[light]{iwona}

% LX fonts
%\usepackage{lxfonts}

% Malvern
%\input T1fmv.fd
%\renewcommand*\sfdefault{fmv}
%\renewcommand*\familydefault{\sfdefault}

% Comfortaa
%\usepackage[default]{comfortaa}

%\setbeamercolor{frametitle}{fg=NavyBlue}
%\setbeamercolor{structure}{fg=NavyBlue}
%\setbeamercolor{normal text}{fg=black}
\setbeamercolor{alerted text}{fg=NavyBlue}
%\setbeamercolor{example text}{fg=red}

\newcommand{\cl}[1]{\multicolumn{1}{c}{#1}}

\newenvironment{changemargin}[2]{%
  \begin{list}{}{%
    \setlength{\topsep}{0pt}%
    \setlength{\leftmargin}{#1}%
    \setlength{\rightmargin}{#2}%
    \setlength{\listparindent}{\parindent}%
    \setlength{\itemindent}{\parindent}%
    \setlength{\parsep}{\parskip}%
  }%
\item[]}{\end{list}}

\begin{document}
\tikzstyle{every picture}+=[remember picture]
\tikzstyle{na} = [baseline=-.5ex]

\begin{frame}
\title{Formación y Evolución de las Galaxias}
%\subtitle{}
\author{Alfredo J. Mej\'ia$^{1,2}$}

\date{\today}

\institute{$^{1}$Posgrado de F\'isica Fundamental\\ Universidad de Los Andes \and $^{2}$Centro de %
Investigaciones de Astronom\'ia%
}

\maketitle
\end{frame}

%Motivación
\begin{frame}[allowframebreaks]{\textsc{Motivación}}
%
Si bien los sondeos de galaxias de última generación han revelado detalles de los fenómenos físicos
que intervienen en la formación y evolución de las galaxias, una construcción física \emph{ab
initio} permanece aún elusiva. La principal limitación en este sentido sigue siendo la tecnología.
Por otra parte, desde el punto de vista teórico, mediante simulaciones cosmológicas autoconsistentes
de formación de estructuras a gran escala, hemos llegado a la satisfactoria realización de que
conocemos (y hasta cierto punto, entendemos) los aspectos físicos más relevantes en la formación de
las galaxias. Existen esencialmente dos formas de modelar los procesos físicos de la formación y la
evolución de las galaxias en dichas simulaciones: uno es mediante simulaciones hidrodinámicas de
materia oscura y bariónica y el otro es el llamado método semianalítico; aunque ambos métodos son
fundamentalmente distintos, obtienen resultados similares. Más aún, estos están en acuerdo
cualitativo con las observaciones. Aún así, existen fenómenos físicos de gran importancia en la
construcción de una teoría de formación de galaxias, que carecen aún de un entendimiento completo.
Es el objeto de este seminario hacer un recuento fenomenológico de los procesos físicos que
intervienen en la formación de las galaxias, señalar las incertidumbres que existe en cada proceso y
establecer las perspectivas a futuro. Ya que el enfoque es fenomenológico, este seminario estará
naturalmente sesgado hacia los resultados del método semianalítico, sin embargo, cuando sea
pertienente señalaré las diferencias entre los resultados de ambos métodos en comparación con las
observaciones.
%
\end{frame}

%Historia
\begin{frame}[allowframebreaks]{\textsc{Antecedentes}}
%
\begin{description}
%
\item[\textsc{Formación de galaxias.}] Aunque en los primeros modelos de formación de galaxias
lograban predecir algunas propiedades globales, como la función de luminosidad de las galaxias y la
existencia de galaxias satelites y galaxias centrales masivas compartiendo el mismo halo de materia
oscura \citep{White1978}. Más aún, el modelo de colapso monolítico aún permitía predecir algunas de
las propiedades observadas en galaxias elípticas \citep{Larson1974a, Larson1974c}. En particular,
estos modelos predecían colores más azules hacia el núcleo galáctico que los observados, esto debido
a que la tasa de formación estelar se extendendía en desde el momento en que el colapso iniciaba
hasta el presente. \citet{Larson1974c} sugería que para `apagar' la formación estelar en este tipo
de galaxias, una fuente de energía en el núcleo de estas galaxias era necesaria para barrer y
calentar el material gaseoso. La realización de que algunos fenómenos físicos aún faltaban en la
construcción de una teoría para describir la formación y evolución de las galaxias, era clara.

Primeros indicios de que un número significativo de galaxias podrían albergar un agujero negro
supermasivo ($\sim10^6\,$---$\,10^9\,$M$_\odot$) apareció entre finales de los 80's y comienzos de
los 90's \citep[véase][para una revisión]{Kormendy1995}. Estos indicios, aunque en su mayoría
circunstanciales, fueron suficientes para motivar la búsqueda de agujeros negros supermasivos en el
universo local. Eventualmente, los movimientos estelares a escalas de unos pocos parsecs proveyó
evidencia irrefutable de que muchas galaxias (independientemente de la clase morfológica),
albergaban un agujero negro supermasivo en su núcleo. Más importante aún, las masas de dicho agujero
negro estaban correlacionadas con la luminosidad \citep{Magorrian1998} y con la masa \citep[medida
con la dispersión de velocidades central][]{Ferrarese2000}. Estas correlaciones indicaban claramente
que debía existir un vínculo fundamental entre la formación de las galaxias y la formación del
agujero negro supermasivo que estas albergan. Eventualmente, los intentos por teorizar la formación
de las galaxias comenzaron a incluir la formación y evolución de agujeros negros supermasivos y los
efectos ambientales que estos producían \citep[e.\,g.][]{Springel2005d}. La lista de fenómenos
físicos que capturaban la esencia de la formación y evolución de galaxias en el universo tomó la
forma que hoy conocemos:
%
\begin{itemize}
\item Fusión de (sub)halos (gravedad),
\item enfriamiento del material bariónico (termodinámica y transferencia radiativa),
\item formación estelar,
\item evolución química,
\item \emph{feedback} de la formación estelar,
\item formación de agujeros negros supermasivos, y
\item \emph{feedback} de núcleos activos.
\end{itemize}
%
\item[\textsc{Predicciones.}] Los primeros intentos por simular la formación de estructuras a
escalas cosmológicas, lograban hacer predicciones cualitativamente en acuerdo un conjunto de
observaciones:
%
\begin{itemize}
%
\item La formación de estructuras sobredensas,
\item la formación de cúmulos de galaxias,
\item los propiedades cinemáticas de esferoides y discos estelares,
%
\end{itemize}
%
Sin embargo fallaron en predecir propiedades observadas directamente como los colores y otras
estimadas como la tasa de formación estelar.
%
\end{description}

% \begin{itemize}
% \item Primeros intentos por teorizar la formación de las galaxias.
% \item ¿Cuáles eran los ingredientes físicos?.
% \item ¿Qué lograban predecir?.
% \end{itemize}
%
\end{frame}

\begin{frame}[allowframebreaks]{\textsc{Teoría de formación de galaxias -- en construcción}}
%
En el marco del \emph{Big Bang} se postula que el universo comenzó en un estado altamente denso,
caliente y esencialmente homogéneo. Durante un corto período en su expansión, llamado inflación, se
manifestaron fluctuaciones cuánticas que dieron origen a inhomogeneidades, las mismas que detectamos
ahora en el Fondo Cósmico de Microondas (FCM). Las observaciones del FCM junto con mediciones de la
distancia y las Oscilaciones Acústicas de Bariones (OAB) permiten determinar los parámetros
cosmológicos con un nivel de incertidumbre considerablemente bajo ($\sim10\,$por ciento). Estas
mediciones han permitido determinar que el universo en el presente es plano y que está dominado por
materia oscura y energía oscura en más de un $95\,$por ciento. Aunque no sabemos qué es exactamente
la materia oscura, simulaciones han permitido descartar entre un gran número de candidatos. Es
comúnmente asumido que la materia oscura se comporta como un fluído frío y no colisional y
constituye el $\sim25\,$por ciento de la materia-energía en el universo. La energía oscura, por otra
parte, es aún más misteriosa y se ha introducido como un parámetro \emph{ad hoc} para explicar la
geometría y la tasa de expansión (acelerada) del universo. En las ecuaciones fundamentales de la
Relatividad General, la energía oscura toma la forma de una constante, llamada Constante Cosmológica
$\Lambda$ y representa el $\sim70\,$por ciento de la materia-energía. El restante $\sim4\,$por
ciento es materia bariónica.

Dadas estas condiciones iniciales, en un universo en expansión que contiene solo materia oscura, se
desarrollan de manera natural regiones de sobre densidad. A medida que el universo se expande la
densidad de campo (densidad promedio) disminuye. Sin embargo, cuando en estas regiones de sobre
densidad se alcanza un valor crítico, la expansión del universo es despreciable y el material
acumulado se convierte en autogravitante.

Los sondeos del cielo de los últimos 20 años nos ha permitido estudiar con relativo detalle (aún así
estadísticamente hablando) los propiedades observables y físicas de las galaxias en escalas de
tiempo cosmológicas ($z<6$). Por una parte los métodos que buscan explotar los registros fósiles en
el universo local ($z<0.5$) han servido su propósito revelando correlaciones entre las propiedades
físicas de las galaxias (\cites[la relación masa-metalicidad,][]{Tremonti2004, Sanchez2013}[la
relación edad-metalicidad,][]{Worthey1994, Gallazzi2005, Panter2008}), mientras que estudios fuera
del universo local ($z>1$) han permitido muestrear la distribución de masa estelar a escalas
cosmológicas, la densidad de TFE, entre otros \citep[véase][para una revisión completa]{Madau2014}.
Estas observaciones y mediciones han permitido la construcción de calibraciones en vista de una
construcción \emph{ab initio} de la formación de galaxias. También nos han permitido decir con
certeza que hoy tenemos conocimiento de los fenómenos físicos que intervienen en la construcción de
dicha teoría.

Aún así, los aspectos fundamentales relacionados con la formación estelar no están completamente
entendidos. Por un lado las simulaciones que apuntan a resolver el problema de la formación de
galaxias a escalas cosmológicas, de primeros principios, (e.\,g.,
\cites[\textsc{millenium}][]{Springel2005c}[\textsc{illustris}][]{Vogelsberger2014}) no han
alcanzado la resolución espacial ($<20\,$pc) y ni en masa ($<10^6$M$_\odot$) requeridos para
experimentar distintos escenarios de formación estelar físicamente plausibles, de nuevo, en el
contexto cosmológico. Por otro lado, las observaciones, incluso en nuestra propia galaxia
\citep[e.\,g.][]{Lada2010}, son aún insuficientes para elaborar una compresión completa de los
mecanismos físicos que intervienen en el MIE y como estos propician o inhiben la formación estelar
\citep[véase][para revisiones recientes]{McKee2007, Krumholz2014, Naab2016}.

A continuación una representación esquemática de los ingredientes necesarios para la elaboración de
una teoría de formación de galaxias.
%
% - Eje horizontal: tiempo cosmico, eje vertical: espacio.
% - Evolución desde la fusión de halos de materia oscura hasta la formación de una galaxia como la
%   veríamos a z=0.
% - En cada estadío evolutivo, representar lo que sucede a distintas escalas espaciales.
%
% \begin{itemize}
% \item Fundamento físico.
% \item Fenómenos que sabemos intervienen.
% \item Resultados.
% \end{itemize}
%
\end{frame}

\begin{frame}[allowframebreaks]{\textsc{Enfriamiento del gas}}
%
% Electron-ionized plasmas (also called collisionally ionized plasmas) are formed in a di-
% verse variety of objects in the universe. These range from stellar coronae and supernova
% remnants to the interstellar medium and gas in galaxies or in clusters of galaxies. The phys-
% ical properties of these sources can be determined using spectral observations coupled with
% theoretical models. This allows one to infer electron and ion temperatures, densities, emis-
% sion measure distributions, and ion and elemental abundances. Reliably determining these
% properties requires accurate fractional abundance calculations for the different ionization
% stages of the various elements in the plasma (i.e., the ionization balance of the gas).
% Since many of the observed sources are not in local thermodynamic equilibrium, in order
% to determine the ionization balance of the plasma one needs to know the rate coefficients
% for all the relevant ionization and recombination processes. Often the observed systems
% are optically-thin, low-density, dust-free, and in steady-state or quasi-steady-state. Under
% these conditions the effects of any radiation field can be ignored, three-body collisions are
% unimportant, and the ionization balance of the gas is time-independent. This is commonly
% called collisional ionization equilibrium (CIE) or sometimes coronal equilibrium.
%
El gas en el halo se asume que es calentado por ondas de choque producidas por el durante el
colapso. Tradicionalmente se asume que el gas se encontraba en equilibrio colisional \citep[sin
embargo véase][]{Wiersma2009a}, por lo tanto el tiempo de enfriamiento se puede escribir como
%
$$
t_\text{enf} = \frac{3}{2}\frac{1}{\mu\,m_\text{H}}\frac{k\,T_\text{gas}}{\rho_\text{gas}(r)\Lambda(T_\text{gas}, Z_\text{gas})}.
$$
%
Bajo esta suposición, el mecanismo de enfriamiento más eficiente a escalas cosmológicas son los
procesos radiativos de dos cuerpos. Dependiendo de la temperatura del gas, el mecanismo específico
de enfriamiento, encapsulado en la función $\Lambda(T_\text{gas}, Z_\text{gas})$, será uno u otro:
%
\begin{description}
%
\item[\textsc{\emph{Bremsstrahlung}.}] A temperaturas $T>10^7\,$K, el gas está completamente
ionizado, por lo tanto se enfría vía interacciones libre-libre como enfriamiento `Bremsstrahlung'.
% mostrar un diagrama representando este proceso.
\item[\textsc{Recombinación.}] En el rango de temperaturas $10^4<T<10^7\,$K, las especies
colisionalmente ionizadas pueden decaer al nivel base y los electrones recombinarse con los iones.
%
\item[\textsc{Emisión de metales.}] A temperaturas $T<10^4\,$K el enfriamiento ocurre mediante
(des)excitación colisional de especies pesadas, mediante un proceso llamado emisión de metales.
%
\end{description}
%
Los mecanismos de enfriamiento dependen, por supuesto, de las condiciones del gas y del medio en que
este se encuentra. Si alguno de los procesos antes mencionados no es capaz de enfriar eficientemente
el gas, este se convertirá en una estructura cuasi-estática soportada por presión, con una
temperatura cercana a la temperatura virial. Eventualmente se enfriará, perderá su estructura
soportada por presión. El gas frío es acretado en un disco en el centro del halo de materia oscura.
Durante el proceso de acreción del gas se asume que el disco conserva su momento angular
\citep[véase][]{White1991, Cole2000}.

Para que la formación estelar tenga lugar, el gas debe entonces enfriarse lo suficiente como para
condensarse en nubes de unas pocas decenas de parsecs.
%
\end{frame}

\begin{frame}[allowframebreaks]{\textsc{Formación estelar y \emph{mergers}}}
%
El contenido estelar de una galaxia se ve afectado principalmente por dos fenómenos: la formación
estelar, que es un fenómeno local que ocurre a escalas de unas pocas decenas de parsecs y la fusión
de halos de materia oscura por acción de la gravedad, el cual tiene lugar en escalas de millones de
parsecs. Las escalas temporales también son bastente diferentes entre un fenómeno y otro. La
formación estelar, bajo la suposición de que no existe \emph{feedback} estelar o de AGN (los cuales
explicaré más adelante), es un fenómeno secular, es decir, se puede extender desde la formación del
disco hasta el presente. El fenómeno de fusión de halos, por otra parte, puede borrar sus huellas en
cuestión de unos pocos giga años.

\begin{description}
%
\item[\textsc{Formación estelar.}] Formación estelar es probablemente el fenómeno físico más
fundamental en la astrofísica. Parte de lo que lo hace interesante es que aún no están resueltas las
condiciones físicas bajo las cuales tiene lugar, pues las propiedades físicas del MIE (donde la
formación estelar tiene lugar) son extremadamente complejas: primero el MIE está compuesto por gas
en distintas fases coexistiendo en un amplio rango de densidades y temperaturas. Desde el punto de
vista de las simulaciones \emph{ab initio} a gran escala (e.\,g. simulaciones cosmológicas) estas
condiciones físicas aún están lejos de ser resueltas. Sin embargo, la existencia de relaciones
`globales' o que se manifiestan de forma global en poblaciones de galaxias nos ha permitido hacer
avances en nuestro entendimiento teórico de los procesos físicos involucrados en la formación
estelar. En este sentido, las simulaciones \emph{ab initio} y las semi-analíticas han desarrollado
recetas con la sola meta de predecir dichas relaciones, es decir, con la sola meta de predecir de
manera acertada dichas relaciones.

En una visión bastante esquemática, el proceso de formación estelar, partiendo del acentamiento del
gas frío en una estructura soportada por momento angular, se puede resumir como sigue:
%
\begin{itemize}
\item Formación de gas atómico frío,
\item formación de nubes de gas auto-gravitantes,
\item formación de moleculas y nubes moleculares,
\item formación de aglomeraciones densas dentro de esas nubes,
\item formación de núcleos preestelares, estrellas y cúmulos estelares.
\end{itemize}
%
De este conjunto de eventos, podemos decir que conocemos con cierto grado de certidumbre las
condiciones iniciales, es decir, la formación del disco de gas, y el estadío final, cuando las
estrellas y los cúmulos estelares ya se han formado. Desafortunadamente, las propiedades ópticas del
del MIE en nubes de formación estelar dificultan la resolución de los fenómenos físicos mediante
estudios observacionales. En particular, la formación de gas atómico, gas molecular y la formación
subsiguiente de nubes moleculares, e incluso su rol en la formación estelar, son aún muy incertas.
Multiples mecanismos siendo candidatos plausibles para la activación de la formación estelar una vez
el gas se ha acentado en el disco. Desde el punto de vista teórico, en las simulaciones de formación
de galaxias es común parametrizar la tasa de formación estelar,
%
$$
\dot{\rho}_\star = \epsilon_\star\,\frac{\rho_\text{gas}}{t_\star}\propto{\rho_\text{gas}}^{1.5},
$$
%
e imponer un conjunto de reglas (condiciones) físicas para su activación. Una de estas reglas,
y probablemente la más fundamental, es la escala de tiempo, $t_\star$, en que la formación estelar
ocurre localmente. En general esta escala temporal depende de las propiedades físicas del gas y
puede ser función de escala dinámica de tiempo, del tiempo de enfriamiento, de la escala de tiempo
para la formación de moleculas (e.\,g. H$_2$). Ciertamente, aún no está claro si alguno de los
procesos físicos que involucra adoptar una escala temporal u otra, es preponderante en la formación
estelar y bajo qué circunstancias lo es.

Uno puede acotar la física que interviene en la formación estelar suponiendo dos esquemas: uno en el
que los efectos locales dominan los mecanismos de formación estelar y otro en el que los efectos
globales lo hacen. En el primer esquema se asume que la formación estelar es controlada en el
corazón de nubes moleculares, en particular, por su estructura y abundancia; una afirmación que es
respaldada por evidencia observacional que conduce a una correlación entre la TFE y la cantidad de
gas denso en nubes moleculares \citep[e.\,g.][]{Lada2010}. \citet{Lada2010} encontraron que existe
una fuerte correlación entre la abundancia de gas molecular en regiones de alta densidad con el
número de objetos estelares jóvenes (directamente proporcional a la TFE). Bajo este esquema, la
relación entre la TFE y la densidad del gas se explica por un decremento en la escala temporal o por
un incremento de la fracción de gas por encima de cierto umbral, suponiendo $t_\star=f\times
t_\text{cl}$.
% mostrar el gráfico 4 de Lada+2010.
Bajo el segundo esquema, la formación estelar es dominada globalmente por fenómenos dinámicos, en
lugar de los mecanismos de enfriamiento y la formación de gas molecular. Independientemente de la
fase del gas, atómico o molecular, la densidad superficial de este es lo que controla la TFE. En
lugar de suponer umbrales relacionados con el enfriamiento y la formación de gas molecular, este
esquema supone que la transición entre los distintos regímenes de formación estelar está relacionada
con inestabilidades dinámicas en el disco. En este sentido, la relación entre la TFE y la densidad
del gas surge por escalas temporales acortadas en regiones de alta densidad o escalas dinámicas más
cortas \citep[e.\,g.][]{Ostriker2010}.
%
El esquema local explica naturalmente las relaciones locales entre las propiedades del gas molecular
y la TFE. Sin embargo no explica por qué una galaxia o parte de ella se encuentra en distintos
regímenes de formación estelar. El esquema global, por otro lado, sugiere que los modos de formación
estelar relacionados con regiones de baja densidad pueden explicarse por inestabilidades dinámicas,
una afirmación que es aún incierta. A altas densidades las altas TFE se explican por el
compactamiento del gas. La eficiencia es regulada por el \emph{feedback} estelar, sin embargo, la
observación de que la eficiencia es similar en nubes moleculares independientemente de que estas
estén formando estrellas masivas o no, plantea dudas sobre este esquema.
%
\item[\textsc{Fusión de halos.}] La fusión de halos o \emph{merger} es uno de los mecanismos
mediante los cuales la masa luminosa de una galaxia aumenta. Por supuesto, si una de las galaxias en
el merger contiene una fracción importante de gas, la fusión o incluso la interacción entre las
galaxias puede propiciar eventos de formación estelar violentos, caracterizados por un incremento en
la eficiencia $\epsilon_\star$. La fusión de galaxias puede también cambiar dramáticamente la
estructura de las galaxias y propiciar el acrecimiento de materia al agujero negro supermasivo en la
región central de las galaxias. Las simulaciones de fusión de halos de materia oscura (junto con la
galaxia correspondiente), han permitido avances significativos hacia la elaboración de una teoría
que explique la formación de galaxias y su evolución subsiguiente. Entre las conclusiones a las que
hemos llegado están \citep{Naab2016}:
%
\begin{itemize}
\item La fusión binaria de halos es poco frecuente,
\item en galaxias de baja masa y masa intermedia las estrellas se forman \emph{in situ},
\item en galaxias masivas la fusión de galaxias es importante,
\end{itemize}
%
Uno puede distinguir entre dos casos extremos en la fusión de halos. Uno en el que la componente
gaseosa es despreciable y por lo tanto la interacción es no colisional, y otro donde la componente
gaseosa es importante.

En el primer caso solo materia oscura y materia luminosa interactúan gravitacionalmente y
dependiendo de la relación entre las masas, la galaxia resultante puede desarrollar un halo más
compacto debido a relajación violenta. El proceso de relajación violenta propiciará un intercambio
de momento angular entre el disco de la componente principal y el halo correspondiente, produciendo
un sistema cinemáticamente caliente. En general existirá transferencia de momento angular entre el
halo y la componente estelar y entre la componente estelar siendo engullida y la componente
existente. El momento angular orbital de la componente de menor masa se transformará en momento
rotacional en la componente más masiva, de manera que la galaxia resultante rotará más que sus
progenitoras, independientemente de la morfología de las componentes interactuantes \citep{Qu2017}.

Si los sistemas interactuantes son esferoidales (y una es mucho menos masiva que la otra), no habrá
relajación violenta y la galaxia de mayor masa asimilará a la galaxia más pequeña en un radio mayor.
Este mecanismo, de hecho es capaz de explicar con éxito la evolución estructural de las galaxias
tempranas más masivas que vemos en el universo local. Sin embargo, el rol de las fusiones menores en
la formación de este tipo de galaxias dependerá de la frecuencia de estos eventos y de la morfología
de las componentes de involucradas. Si la frecuencia de fusiones es muy baja, por ejemplo, la
componente más masiva podría conservar su morfología original. En el caso de un disco estelar,
podría haber un calentamiento moderado.

Por supuesto cuando la componente gaseosa no es despreciable en ninguna de las galaxias
interactuantes ya no es posible asumir dinámica no colisional, tampoco es posible ignorar los
efectos de transporte de energía mediante radiación. Interacciones de este tipo parecen ser las
responsables de los fenómenos de brotes de formación estelar, como los observados en galaxias
ultra-luminosas en el IR. El gas acentado en el centro de la galaxia puede hacer que el potencial
sea más esférico, propiciando el desarrollo de cinemática rotacional y galaxias con morfología de
disco \citep{Jesseit2007}. En este tipo de interacciones, la componente gaseosa acretada por el
disco, si tiene momento angular lo suficientemente bajo puede ser acretada por el agujero negro
supermasivo central de la galaxia. La energía liberada durante el proceso de acreción se ha
encontrado responsable de imprimir momentum al gas acentado en el disco, disminuyendo dramáticamente
la formación estelar (más adelante explicaré con más detalle la física detrás de los procesos de
\emph{feedback}).
% Diagrama resumen de las implicaciones de los procesos de fusión de galaxias.
%
\end{description}
%
\end{frame}

\begin{frame}[allowframebreaks]{\textsc{Evolución química}}
%
Una de las consecuencias de la evolución estelar es el enriquecimiento químico del MIE y del MIG.
Posteriormente parte del material devuelto al MIG en eventos de supernovas y/o AGNs puede retornar
al MIE, en el que el ciclo de formación estelar tiene lugar. La evolución química de estos medios es
importante para la formación y evolución de galaxias por varias razones:
%
\begin{itemize}
\item La tasa de enfriamiento del gas puede depender de la metalicidad,
\item la presencia de elementos pesados en el MIE propicia el crecimiento de granos de polvo,
\item las propiedades observables de las galaxias son afectadas directa e indirectamente.
\end{itemize}
%
El objetivo principal de la teoría de evolución química del material bariónico es seguir la
evolución de la abundancia de las distintas especies \emph{observadas} en estrellas y en el MIE/MIG.
El punto de partida es el calculo de la masa retornada por las estrellas durante su vida al MIE.
Existen principalmente dos mecanismos mediante los cuales las estrellas pueden retornar material al
MIE: vientos estelares y explosiones de supernova. El primer mecanismo es importante para el retorno
al MIE del material enriquecido producido en el proceso de nucleosíntesis en el interior de las
estrellas y es particularmente eficiente en la fase de gigantes de las estrellas de masa intermedia,
$\lesssim5\,$M$_\odot$. En estrellas de mayor masa $<100\,$M$_\odot$, los vientos estelares y la
nucleosíntesis en explosiones de supernova al final de la vida de estas estrellas son más
importantes. Por otra parte, existen especies en la composición inicial de las estrellas que no es
ni creado ni destruido durante la vida de las estrellas. La importancia de estas especies dependerá
por supuesto de la composición química inicial de las estrellas. La masa retornada por una estrella
en un momento dado al MIE depende entonces de varios factores:
%
\begin{itemize}
\item la masa inicial de la estrella,
\item La composición química inicial de la estrella,
\item la edad de la estrella.
\end{itemize}
%
Sin embargo, las propiedades estelares son descritas más facilmente si se considera que las
estrellas nacen en grupo a partir del material en nubes moleculares, por lo tanto es una
aproximación justa asumir que en un instante dado un conjunto de estrellas tendrá la misma
composición química, $Z_\star$ y la misma edad, $t$ y estará distribuída en masa de acuerdo a una
FIM, $\phi(m,Z_\star)$. El artificio físico que mejor describe a tal población es conocido como una
población estelar simple (PES). La tasa de masa eyectada por una PES en el instante $t$ es
\citep{Yates2013}:
%
$$
e_\text{masa}(t) = \int_{m_\text{inf}}^{m_\text{sup}}\left[m-m_\text{rem}(m,Z_\star)\right]\psi(t-\tau(m,Z_\star))\phi(m,Z_\star)\text{d}m,
$$
%
donde $m_\text{rem}$ es la masa del remanente estelar que deja una estrella de masa y metalicidad
iniciales $m$ y $Z_\star$, respectivamente; $\tau(m,Z_\star)$ es el tiempo de vida y
$\phi(m,Z_\star)$ es la FIM normalizada de manera que su integral en el rango
$m_\text{inf}\,$---$\,m_\text{sup}$ es igual al número de estrellas en una PES de $1\,$M$_\odot$. En
este sentido, el producto $\psi(t-\tau)\times\phi$ es la tasa de nacimiento de estrellas de masa $m$
y metalicidad $Z$ en el instante $t-\tau$. El producto de esta tasa de nacimiento por
$m-m_\text{rem}$ es simplemente la tasa de eyección instantánea de masa en estrellas de masa $m$ en
$t$. Ahora, más interesante que esto es la masa en \emph{metales} eyectada por una PES:
%
$$
e_Z(t) = \int_{m_\text{inf}}^{m_\text{sup}}\left[y(m,Z_\star)+Z_\star\times(m-m_\text{rem})\right]\psi(t-\tau(m,Z_\star))\phi(m,Z_\star)\text{d}m,
$$
%
donde $y(m,Z_\star)$ es el \emph{yield} (la masa en metales nucleosintéticos) y
$Z_\star\times(m-m_\text{rem})$ es la masa inicial en metales que no fue procesada por la estrella.
La suma de estos dos términos es la masa en metales retornada al MIE por una estrella de masa y
metalicidad iniciales, $m$ y $Z_\star$, respectivamente. Lo que en realidad nos interesa es la
contribución en masa de cada especie. En ese caso el corchete en la integral anterior se puede
escribir como:
%
$$
m_i = y_i(m,Z_\star) + \frac{m_i}{m}(m-m_\text{rem}).
$$
%
Las funciones $\tau(m,Z_\star)$, $y_i(m,Z_\star)$ son resultados de la teoría de evolución estelar
\citep[e.\,g.][]{Portinari1998, Marigo2001}.
% mostrar fig. 1 de Yates et al. 2013 (tiempos de vida).
De acuerdo de predicciones de la teoría de evolución estelar, el \emph{yield} es el resultado de
distintos mecanismo de eyección que se activan en tiempos diferentes, como podemos ver en los
gráficos a continuación \citep{Yates2013}.
% mostrar fig. 2 Yates et al. 2013 (masa de distintas especies eyectada por estrellas en AGB).
% mostrar fig. 3 Yates et al. 2013 (masa de distintas especies eyectada por supernovas Ia).
% mostrar fig. 4 Yates et al. 2013 (masa de distintas especies eyectada por supernovas II).
En la práctica solo un número reducido de especies es considerado puesto que estas son relativamente
más abundantes que otras y por lo tanto más facilmente observables/medibles. En particular las
especies en color en el gráfico son importantes en el calculo de las tasas de enfriamiento del gas
\citep{Wiersma2009a}.
%
\end{frame}

\begin{frame}[allowframebreaks]{\textsc{Fenómenos de \emph{Feedback}}}
%
Los fenómenos de \emph{feedback} (e.\,g., vientos estelares y los eventos de supernova) no solo
pueden cambiar la composición química del MIE, también inyectan cantidades importantes de energía
cinética. Como tal, el efecto neto del \emph{feedback} estelar es el de regular la TFE. Tal
regulación puede ocurrir de dos formas:
%
\begin{description}
\item[\textsc{Preventiva,}] retardando la acreción del gas al MIE,
\item[\textsc{Eyectiva,}] removiendo el gas previamente acretado al MIE.
\end{description}
%
Ambos efectos tienen un impacto relativo dependiendo de la situación de equilibrio dinámico del
sistema en el que actúan. Por ejemplo, en sistemas cinemáticamente calientes, los fenómenos
preventivos son más efectivos; mientras que en sistemas sostenidos por rotación, los fenómenos
eyectivos juegan un papel más importante.

Por su procedencia, los fenómenos de \emph{feedback} se pueden clasificar en \emph{feedback} estelar
y en \emph{feedback} de núcleos activos.
%
\begin{description}
%
\item[\textsc{Estelar.}] En las secciones anteriores ya vimos que los vientos estelares y los
eventos de supernovas regulan la evolución química en el MIE y la tasa de enfriamiento del gas en el
mismo, por lo tanto, regulan vimos que estos fenómenos efectivamente regulan la TFE. Desde el punto
de vista cinemático, estos fenómenos inyectan cantidades importantes de momento y energía necesarios
para regular la TFE en regiones de formación estelar, donde la eficiencia observada es de apenas
$\sim1\,\%$. Más aún la fracción de materia bariónica presente en galaxias se estima que es
$\sim20\,\%$. En efecto, los trabajos en los que se incorporó únicamente el efecto termodinámico del
\emph{feedback} estelar llevaron a la conclusión inmediata de que los fenómenos radiativos
únicamente no era lo suficientemente eficientes como para explicar la ineficiencia en la formación
estelar ni la pequeña fracción de material bariónico observada en las galaxias. De esta forma, el
momento inyectado en el MIE mediante mecanismos estelares puede parametrizarse en función de la
tasa en la que este es inyectado como:
%
\begin{align}
\dot{V} &= \frac{\dot{p}}{M_\star},\quad\text{ó}\nonumber\\
V       &= \frac{\dot{p}}{\dot{M}_\star},\nonumber
\end{align}
%
La primera relación es más apropiada en regiones donde la población se dispersa en escalas de tiempo
más cortos que el tiempo de vida de las estrellas masivas y la segunda es más relevante en regiones
donde la población está en equilibrio estadístico: la tasa a la que nacen las estrellas es la misma
a la que mueren. Entonces, para que el momento inyectado en el MIE pueda suprimir la formación
estelar y transportar material a escalas de cúmulos o asociaciones estelares tiene que ocurrir que
$\dot{V}\times\tau_\text{din}\gtrsim V_\text{local}$. Alternativamente, para que el momento
inyectado en el MIE pueda remover material del MIE y expulsarlo al MIG tiene que ocurrir que
$V\gtrsim V_\text{global}$.

Entre las fuentes de energía--momento lo suficientemente importantes como para regular la TFE a
escalas galácticas están:
%
\begin{itemize}
\item Supernovas ($V_\text{SN}>48\,$km/s),
\item vientos estelares ($V_\text{viento}>140\,$km/s a $Z=Z_\odot$),
\item presión de radiación ($V_L>190\,$km/s$>V_\text{NM}\sim10\,$km/s),
\item rayos cósmicos,
\item fotoionización y calentamiento FUV.
\end{itemize}
%
\citep[ver][para una revisión completa]{Krumholz2014}.
%
\item[\textsc{Núcleos activos.}] Probablemente una de las evidencias más claras de que las galaxias
más masivas ($>10^{11}\,$M$_\odot$) siguen mecanismos formación/evolución diferentes al que siguen
las galaxias menos masivas ($<10^{11}\,$M$_\odot$) se encuentra en la dicotomía observada en sus
clases espectrales: las galaxias masivas son consistentemente las más rojas en el óptico (en reposo)
mientras que las galaxias menos masivas son sistemáticamente más azules. La distribución de color
como función de la luminosidad muestra una clara bimodalidad \citep{Strateva2001, Baldry2004,
Balogh2004}, sin embargo no muestra fuerte dependencia con el ambiente \citep[medido usando la
densidad local, e.\,g.,][]{Balogh2004}. Desde hace algún tiempo ha sido claro que las galaxias más
masivas se forman en regiones preferiblemente más densas, lo cual podría sugerir que mecanismos
dinámicos dominan la formación de estos objetos. Sin embargo, bajo el escenario de que las
componentes que ensamblan a estas galaxias contienen una fracción importante del material bariónico
en forma de gas, no es posible reproducir galaxias tan rojas como las galaxias que observamos en el
universo local. Más aún, no es posible reproducir la dicotomía en la distribución de color la cual
sugiere que si las galaxias azules de alguna forma son progenitoras de las galaxias rojas, entonces
la evolución/transición tiene que ocurrir en una escala de tiempo lo suficientemente corta como para
que la dicotomía.
% mostrar fig. 1 de Balogh+2004
De manera independiente la aparentemente íntima relación entre la masa del agujero negro central y
la dispersión de velocidades del bulbo sugería un vínculo co-evolutivo entre ambas componentes
\citep{Ferrarese2000, Tremaine2002}. La acreción de material en el agujero central puede activar un
AGN no solo es capaz de inyectar energía térmica a escalas galácticas sino también energía y momento
mecánicos. Se piensa que dicha energía-momento asociados son mantenidos hasta que el agujero negro
ha barrido todo el material disponible para acretar. Si dicha hipótesis es cierta, los AGNs pueden
regular o incluso detener completamente la formación estelar en una galaxia.
% mostrar figs. 2 y 3 de Springel+2005a
Estos primeros resultados de \citet{Springel2005a} parecían prometedores, sin embargo, sobrestimaban
artificialmente la influcencia térmica en la reducción de la TFE. Como hemos visto antes, los
fotones pueden distribuir en energía y momento en el MIE. De hecho, la contribución al
\emph{feedback} del AGN puede dividirse en dos componentes: radiativa y mecánica. La primera es
relativamente simple y establece la conexión entre los flujos entrantes y salientes de materia y el
material acretado, el momento y la energía cinética. Suponiendo leyes de conservación estas
relaciones son:
%
\begin{align}
\dot{M}_\text{acr}    &= \dot{M}_\text{ent} - \dot{M}_\text{sal}, \nonumber\\
\dot{E}_\text{viento} &= \epsilon_\text{viento}\dot{M}_\text{acr}c^2 = \frac{1}{2}\dot{M}_\text{sal}\left<v_\text{viento}^2\right>, \nonumber\\
\dot{p}               &= \dot{M}_\text{sal}\left<v_\text{viento}\right>, \nonumber
\end{align}
%
donde la tasa de acreción en la región central alrededor del agujero negro se puede resolver, según
el modelo de acreción de Bondi-Hoyle-Lyttleton, a partir del radio de Bondi
%
$$
R_\text{Bondi} \equiv \frac{2GM_\text{AN}}{c_\text{s}^2 + v_\text{rel}^2},
$$
%
donde $v_\text{rel}$ es la velocidad relativa del gas respecto del AN. La masa acretada por el
agujero negro es
%
$$
\dot{M}_\text{int} = \dot{M}_\text{Bondi} \equiv \frac{4\pi G^2M_\text{AN}^2\rho}{\left(c_\text{s}^2 + v_\text{rel}^2\right)^{3/2}}.
$$
%
De manera que la masa realmente acretada por el AN es
%
$$
\dot{M}_\text{acr} = \dot{M}_\text{ent}\frac{1}{1+\psi_\text{acr}},
$$
%
donde $\psi_\text{acr} = 2\epsilon_\text{viento}c^2/v_\text{viento}^2$ es un parámetro relacionado
con la eficiencia en dicha acreción y por lo general se fija en las simulaciones de formación de
galaxias.
%
% esquema que representa las interacciones descritas por estas ecuaciones:
% - agujero negro central, acreción, radio de Bondi, flujo de entrada y de salida.
%
El espectro de energía de un AGN tiene un pico importante en la región de los rayos X, los cuales
son capaces de impartir momento en el medio. Durante la fase de AGN, las condiciones del MIE son
modificadas drásticamente y la formación estelar es esencialmente suprimida, de manera que las
galaxias crecen casi exclusivamente mediante fusión de halos. Eventualmente el flujo másico de
entrada se iguala con el se salida y la acreción se detiene. En este punto una fracción importante
del gas que estaba originalmente en el MIE ha sido expulsada al MIG y la radiación emitida por el
AGN ha termalizado el ambiente de manera que este tiene una probabilidad muy baja de reingresar al
halo de materia oscura y resumir la formación estelar. La distribución de colores y de masas en el
universo local parece consistente con esta idea.

Las ideas expuestas en el párrafo anterior resuelven varias incógnitas en la formación de galaxias
masivas: ¿por qué son aparentemente más viejas? ¿el fenómeno de \emph{downsizing}? ¿enriquecimiento
químico del MIG?.
%
\end{description}
%
\end{frame}

\begin{frame}[allowframebreaks]{\textsc{Poder predictivo}}
%
En las siguientes secciones mostraré cómo todas las reglas físicas que he expuesto hasta ahora se
compaginan para reproducir las observaciones que hacemos en el universo local y, por lo tanto
proveen una teoría física (aunque incompleta) para describir la formación y la evolución de las
galaxias.
%

\begin{description}
%
\item[\textsc{Comparación con las observaciones.}]

%
\item[\textsc{Principales incertidumbres.}]

%
\end{description}
%
% \begin{itemize}
% \item Comparación de los resultados de las simulaciones con las observaciones.
% \item Discutir las incertidumbres que intervienen cuando se falla en la predicción.
% \item ¿Cómo se puede mejorar?.
% \end{itemize}
%
\end{frame}

\begin{frame}[allowframebreaks]{\textsc{Resumen y perspectivas}}
%
\begin{itemize}
\item Mencionar los fenómenos físicos fundamentales que intervienen en la formación de las galaxias.
\item Mencionar las incertidumbres en cada uno de estos fenómenos.
\item Mencionar resultados más importantes de las simulaciones.
\item Decir cuales son las perspectivas.
\end{itemize}
%
\end{frame}

\begin{frame}[allowframebreaks]{\textsc{Referencias}}
\printbibliography
\end{frame}

\end{document}
