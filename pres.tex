\documentclass[xcolor=dvipsnames,4pt]{beamer}

\setbeamertemplate{navigation symbols}{}
\usefonttheme{professionalfonts}
\useinnertheme{circles}
\usepackage[spanish]{babel}
\usepackage[T1]{fontenc}
\usepackage[utf8]{inputenc}
\usepackage[orientation=portrait,size=custom,width=32,height=18]{beamerposter}

\usepackage[notocbib]{apacite}
\usepackage{multirow}
\usepackage{graphicx}
\usepackage{tikz}

\usecolortheme[named=Black]{structure}

\usetikzlibrary{arrows,shapes,positioning,backgrounds}

% My default font
%\usepackage{newcent}

% Computer Modern Bright font
%\usepackage{cmbright}

% Iwona light
\usepackage[light,math]{iwona}

% LX fonts
%\usepackage{lxfonts}

% Malvern
%\input T1fmv.fd
%\renewcommand*\sfdefault{fmv}
%\renewcommand*\familydefault{\sfdefault}

% Comfortaa
%\usepackage[default]{comfortaa}

%\setbeamercolor{frametitle}{fg=NavyBlue}
%\setbeamercolor{structure}{fg=NavyBlue}
%\setbeamercolor{normal text}{fg=black}
\setbeamercolor{alerted text}{fg=NavyBlue}
%\setbeamercolor{example text}{fg=red}

\newcommand{\cl}[1]{\multicolumn{1}{c}{#1}}

\newenvironment{changemargin}[2]{%
  \begin{list}{}{%
    \setlength{\topsep}{0pt}%
    \setlength{\leftmargin}{#1}%
    \setlength{\rightmargin}{#2}%
    \setlength{\listparindent}{\parindent}%
    \setlength{\itemindent}{\parindent}%
    \setlength{\parsep}{\parskip}%
  }%
\item[]}{\end{list}}

\begin{document}
\tikzstyle{every picture}+=[remember picture]
\tikzstyle{na} = [baseline=-.5ex]

\begin{frame}
\title{La era de la resolución espacial en sondeos de galaxias}
%\subtitle{}
\author{Alfredo J. Mej\'ia$^{1,2}$}

\date{\today}

\institute{$^{1}$Posgrado de F\'isica Fundamental\\ Universidad de Los Andes \and $^{2}$Centro de%
Investigaciones de Astronom\'ia%
}

\maketitle
\end{frame}

%Motivación
\begin{frame}{}
%
Hoy día existen sondeos que permiten resolver espacialmente a las galaxias en el universo local,
pero ¿qué nos ha llevado a idear esta nueva era de la exploración de nuestro universo?

Las galaxias hasta $z\sim1$ pueden resolverse como objetos extendidos en una imagen astronómica, sin
embargo los sondeos espectroscópicos en su mayoría están limitados a una apertura fija y por lo
general integran en una pequeña región de la imagen producida por las galaxias más cercanas debido a
problemas de ingeniería (mostrar una imgen ejemplo de una galaxia del SDSS, con su $z$ y la apertura
correspondiente). La fotometría, por otro lado, permite integrar una imagen producida por una
galaxia en su totalidad, más aún, permite variar la apertura dentro de la cual se mide el brillo
integrado, pues una vez producida la imagen la forma en que esta se estudia no ofrece límites. Sin
embargo, la fotometría tradicional se ha limitado a integrar la imagen en su totalidad o dentro de
una apertura variable, dependiendo del extensión y/o perfil de la fuente que la produce (e.g. radio
Petrosian, el perfil de brillo).

Ahora somos concientes que en la formación y evolución de las galaxias intervienen una variedad de
fenómenos físicos que tienen lugar en distintas escalas temporales y espaciales. Aunque la
disponibilidad de observaciones profundas del cielo han permitido estudiar la evolución de las
galaxias en escalas temporales del orden de la mitad de la edad de universo, apenas estamos rozando
la superficie de lo que podemos hacer con sondeos que ofrecen resolución espacial.

En este seminario les hablaré de lo que hemos conseguido usando los sondeos con unidades de campo
integrado actuales y las perspectivas con los próximos sondeos del mismo tipo.
%
\end{frame}

%Historia
\begin{frame}{}
%
\begin{itemize}
\item Primera idea de resolución espacial.
\item Limitaciones que impiden desarrollarla.
\item Primer sondeo con resolución espacial.
\item Primeros hallazgos.
\end{itemize}
%
\end{frame}

%Unidad de campo integrado (actual)
\begin{frame}{}
%
\begin{itemize}
\item ¿Qué es?
\item ¿Cómo funciona?
\end{itemize}
%
\end{frame}

%Datos producidos
\begin{frame}{}
%
\begin{itemize}
\item ¿Qué tipo de datos produce?
\item ¿Qué estudios permite?
\end{itemize}
%
\end{frame}

%Resultados
\begin{frame}{}
%
\begin{itemize}
\item Estudios poblacionales.
\item Estudios de dinámica.
\item Relaciones de escala.
\item Problemas abiertos.
\end{itemize}
%
\end{frame}

%Unidad de campo integrado (en el futuro)
\begin{frame}{}
%
\begin{itemize}
\item Mejoras respecto a lo anterior
\item Estudios que permitirá
\item Limitaciones
\end{itemize}
%
\end{frame}


%Perspectivas a futuro (SED-fitting wise)
\begin{frame}{}
%
\begin{itemize}
\item Estudios de relaciones de escala como función de la masa
\item Estudios de la conformidad galáctica
\end{itemize}
%
\end{frame}

\end{document}
