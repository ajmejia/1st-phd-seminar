\documentclass[xcolor=dvipsnames,4pt,hyperref={colorlinks,citecolor=black,linkcolor=black,urlcolor=black}]{beamer}


\setbeamertemplate{navigation symbols}{}
\usefonttheme{professionalfonts}
\useinnertheme{circles}
\usepackage[spanish]{babel}
\usepackage[T1]{fontenc}
\usepackage[utf8]{inputenc}
\usepackage[orientation=portrait,size=custom,width=32,height=18]{beamerposter}

\usepackage{multirow}
\usepackage{graphicx}
\usepackage{tikz}

\usecolortheme[named=Black]{structure}

\usetikzlibrary{arrows,shapes,positioning,backgrounds}

\PassOptionsToPackage{%
%backend=biber, % Instead of bibtex
backend=bibtex8,bibencoding=ascii,%
language=auto,%
%style=numeric-comp,%
style=authoryear-comp, % Author 1999, 2010
bibstyle=authoryear,dashed=false, % dashed: substitute rep. author with ---
sorting=ynt, % year, name, title
maxbibnames=3, % default: 3, et al.
backref=false,%
natbib=true, % natbib compatibility mode (\citep and \citet still work)
url=false, %
doi=false, %
eprint=false %
}{biblatex}
\usepackage{biblatex}

\addbibresource{library.bib}

\hypersetup{
% Uncomment the line below to remove all links (to references, figures, tables, etc), useful for b/w printouts
%draft,
colorlinks=true, linktocpage=true, pdfstartpage=3, pdfstartview=FitV,
% Uncomment the line below if you want to have black links (e.g. for printing black and white)
%colorlinks=false, linktocpage=false, pdfborder={0 0 0}, pdfstartpage=3, pdfstartview=FitV,
breaklinks=true, pdfpagemode=UseNone, pageanchor=true, pdfpagemode=UseOutlines,%
plainpages=false, bookmarksnumbered, bookmarksopen=true, bookmarksopenlevel=1,%
hypertexnames=true, pdfhighlight=/O,%nesting=true,%frenchlinks,%
urlcolor=webbrown, linkcolor=RoyalBlue, citecolor=Maroon, %pagecolor=RoyalBlue,%
%urlcolor=Black, linkcolor=Black, citecolor=Black, %pagecolor=Black,%
}

% My default font
%\usepackage{newcent}

% Computer Modern Bright font
%\usepackage{cmbright}

% Iwona light
\usepackage[light,math]{iwona}

% LX fonts
%\usepackage{lxfonts}

% Malvern
%\input T1fmv.fd
%\renewcommand*\sfdefault{fmv}
%\renewcommand*\familydefault{\sfdefault}

% Comfortaa
%\usepackage[default]{comfortaa}

%\setbeamercolor{frametitle}{fg=NavyBlue}
%\setbeamercolor{structure}{fg=NavyBlue}
%\setbeamercolor{normal text}{fg=black}
\setbeamercolor{alerted text}{fg=NavyBlue}
%\setbeamercolor{example text}{fg=red}

\newcommand{\cl}[1]{\multicolumn{1}{c}{#1}}

\newenvironment{changemargin}[2]{%
  \begin{list}{}{%
    \setlength{\topsep}{0pt}%
    \setlength{\leftmargin}{#1}%
    \setlength{\rightmargin}{#2}%
    \setlength{\listparindent}{\parindent}%
    \setlength{\itemindent}{\parindent}%
    \setlength{\parsep}{\parskip}%
  }%
\item[]}{\end{list}}

\begin{document}
\tikzstyle{every picture}+=[remember picture]
\tikzstyle{na} = [baseline=-.5ex]

\begin{frame}
\title{Formación y Evolución de las Galaxias}
%\subtitle{}
\author{Alfredo J. Mej\'ia$^{1,2}$}

\date{\today}

\institute{$^{1}$Posgrado de F\'isica Fundamental\\ Universidad de Los Andes \and $^{2}$Centro de %
Investigaciones de Astronom\'ia%
}

\maketitle
\end{frame}

%Motivación
\begin{frame}[allowframebreaks]{\textsc{Motivación}}
%
Si bien los sondeos de galaxias de última generación han revelado detalles de los fenómenos físicos
que intervienen en la formación y evolución de las galaxias, una construcción física \emph{ab
initio} permanece aún elusiva. La principal limitación en este sentido sigue siendo la tecnología.
Por otra parte, desde el punto de vista teórico, mediante simulaciones cosmológicas autoconsistentes
de formación de estructuras a gran escala, hemos llegado a la satisfactoria realización de que
conocemos (y hasta cierto punto, entendemos) los aspectos físicos más relevantes en la formación de
las galaxias. Existen esencialmente dos formas de modelar los procesos físicos de la formación y la
evolución de las galaxias en dichas simulaciones: uno es mediante simulaciones hidrodinámicas de
materia oscura y bariónica y el otro es el llamado método semianalítico; aunque ambos métodos son
fundamentalmente distintos, obtienen resultados similares. Más aún, estos están en acuerdo
cualitativo con las observaciones. Aún así, existen fenómenos físicos de gran importancia en la
construcción de una teoría de formación de galaxias, que carecen aún de un entendimiento completo.
Es el objeto de este seminario hacer un recuento fenomenológico de los procesos físicos que
intervienen en la formación de las galaxias, señalar las incertidumbres que existe en cada proceso y
establecer las perspectivas a futuro. Ya que el enfoque es fenomenológico, este seminario estará
naturalmente sesgado hacia los resultados del método semianalítico, sin embargo, cuando sea
pertienente señalaré las diferencias entre los resultados de ambos métodos en comparación con las
observaciones.
%
\end{frame}

%Historia
\begin{frame}[allowframebreaks]{\textsc{Antecedentes}}
%
\begin{description}
%
\item[\textsc{Formación de galaxias.}] Es comúnmente aceptado (y físicamente plausible) que en un
universo en expansión acelerada, dominado por materia oscura fría las galaxias se forman en regiones
de sobredensidad que llamamos halos de materia oscura \citep{Davis1985}. Estos halos pueden, por
acción de la gravedad, fusionarse con halos vecinos para dar origen a galaxias más masivas. Este
proceso es denominado formación formación jerárquica de galaxias \citep{Baugh1996, Kauffmann1996}.
Un conjunto de procesos físicos que tienen lugar a diferentes escalas espaciales
($\sim10\,$---$\,10^6\,$pc) y temporales ($\sim10^6\,$---$\,10^9\,$años) regulan la subsiguiente
evolución de las galaxias. En los primeros intentos por teorizar dicho proceso de formación y
evolución de las galaxias en el marco de la cosmología moderna, se identificaron los siguiente
procesos físicos como fundamentales:
%
\begin{itemize}
\item Fusión de (sub)halos,
\item enfriamiento del material bariónico (termodinámica y transferencia radiativa),
\item formación estelar,
\item evolución química, y
\item \emph{feedback} de la formación estelar \citep{Larson1974a, Larson1974b, White1978}.
\end{itemize}
%
Aunque en los primeros modelos de formación de galaxias lograban predecir algunas propiedades
globales, como la función de luminosidad de las galaxias y la existencia de galaxias satelites y
galaxias centrales masivas compartiendo el mismo halo de materia oscura \citep{White1978}. Más aún,
el modelo de colapso monolítico aún permitía predecir algunas de las propiedades observadas en
galaxias elípticas \citep{Larson1974a, Larson1974c}. En particular, estos modelos predecían colores
más azules hacia el núcleo galáctico que los observados, esto debido a que la tasa de formación
estelar se extendendía en desde el momento en que el colapso iniciaba hasta el presente.
\citet{Larson1974c} sugería que para `apagar' la formación estelar en este tipo de galaxias, una
fuente de energía en el núcleo de estas galaxias era necesaria para barrer y calentar el material
gaseoso. La realización de que algunos fenómenos físicos aún faltaban en la construcción de una
teoría para describir la formación y evolución de las galaxias, era clara.

Primeros indicios de que un número significativo de galaxias podrían albergar un agujero negro
supermasivo ($\sim10^6\,$---$\,10^9\,$M$_\odot$) apareció entre finales de los 80's y comienzos de
los 90's \citep[véase][para una revisión]{Kormendy1995}. Estos indicios, aunque en su mayoría
circunstanciales, fueron suficientes para motivar la búsqueda de agujeros negros supermasivos en el
universo local. Eventualmente, los movimientos estelares a escalas de unos pocos parsecs proveyó
evidencia irrefutable de que muchas galaxias (independientemente de la clase morfológica),
albergaban un agujero negro supermasivo en su núcleo. Más importante aún, las masas de dicho agujero
negro estaban correlacionadas con la luminosidad \citep{Magorrian1998} y con la masa \citep[medida
con la dispersión de velocidades central][]{Ferrarese2000}. Estas correlaciones indicaban claramente
que debía existir un vínculo fundamental entre la formación de las galaxias y la formación del
agujero negro supermasivo que estas albergan. Eventualmente, los intentos por teorizar la formación
de las galaxias comenzaron a incluir la formación y evolución de agujeros negros supermasivos y los
efectos ambientales que estos producían \citep[e.\,g.][]{Springel2005}:
%
\begin{itemize}
\item Formación de agujeros negros supermasivos, y
\item \emph{feedback} de núcleos activos.
\end{itemize}
%
\item[\textsc{Predicciones.}] Los primeros intentos por simular la formación de estructuras a
escalas cosmológicas, lograban hacer predicciones cualitativamente en acuerdo un conjunto de
observaciones:
%
\begin{itemize}
%
\item La formación de estructuras sobredensas,
\item la formación de cúmulos de galaxias,
\item los propiedades cinemáticas de esferoides y discos estelares,
%
\end{itemize}
%
Sin embargo fallaron en predecir propiedades observadas directamente como los colores y otras
estimadas como la tasa de formación estelar.
%
\end{description}

% \begin{itemize}
% \item Primeros intentos por teorizar la formación de las galaxias.
% \item ¿Cuáles eran los ingredientes físicos?.
% \item ¿Qué lograban predecir?.
% \end{itemize}
%
\end{frame}

\begin{frame}[allowframebreaks]{\textsc{Teoría de formación de galaxias}}
%
Ahora podemos decir que somos conscientes de los fenómenos más importantes que intervienen en la
formación de estructuras a gran escala en el universo.
%
\begin{itemize}
\item Fundamento físico.
\item Fenómenos que sabemos intervienen.
\item Resultados.
\end{itemize}
%
\end{frame}

%
\begin{frame}[allowframebreaks]{\textsc{Simulaciones hidrodinámicas}}
%
\begin{itemize}
\item Física de entrada.
\item Procedimientos.
\item Resultados.
\end{itemize}
%
\end{frame}

\begin{frame}[allowframebreaks]{\textsc{Simulaciones semianalíticas}}
%
\begin{itemize}
\item Física de entrada.
\item Procedimientos.
\item Resultados.
\end{itemize}
%
\end{frame}

\begin{frame}[allowframebreaks]{\textsc{Análisis comparativo}}
%
\begin{itemize}
\item Fortalezas y debilidades de ambos métodos.
\item Perspectivas.
\end{itemize}
%
\end{frame}

\begin{frame}[allowframebreaks]{\textsc{Poder predictivo}}
%
\begin{itemize}
\item Comparación de los resultados de las simulaciones con las observaciones.
\item Discutir las incertidumbres que intervienen cuando se falla en la predicción.
\item ¿Cómo se puede mejorar?.
\end{itemize}
%
\end{frame}

\begin{frame}[allowframebreaks]{\textsc{Resumen y perspectivas}}
%
\begin{itemize}
\item Mencionar los fenómenos físicos fundamentales que intervienen en la formación de las galaxias.
\item Mencionar las incertidumbres en cada uno de estos fenómenos.
\item Mencionar resultados más importantes de las simulaciones.
\item Decir cuales son las perspectivas.
\end{itemize}
%
\end{frame}

\begin{frame}[allowframebreaks]{\textsc{Referencias}}
\printbibliography
\end{frame}

\end{document}
