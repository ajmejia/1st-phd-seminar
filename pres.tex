\documentclass[xcolor=dvipsnames,4pt,hyperref={colorlinks,citecolor=black,linkcolor=black,urlcolor=black}]{beamer}


\setbeamertemplate{navigation symbols}{}
\usefonttheme{professionalfonts}
\useinnertheme{circles}
\usepackage[spanish]{babel}
\usepackage[T1]{fontenc}
\usepackage[utf8]{inputenc}
\usepackage[orientation=portrait,size=custom,width=32,height=18]{beamerposter}

\usepackage{multirow}
\usepackage{graphicx}
\usepackage{tikz}

\usecolortheme[named=Black]{structure}

\usetikzlibrary{arrows,shapes,positioning,backgrounds}

\PassOptionsToPackage{%
%backend=biber, % Instead of bibtex
backend=bibtex8,bibencoding=ascii,%
language=auto,%
%style=numeric-comp,%
style=authoryear-comp, % Author 1999, 2010
bibstyle=authoryear,dashed=false, % dashed: substitute rep. author with ---
sorting=ynt, % year, name, title
maxbibnames=3, % default: 3, et al.
backref=false,%
natbib=true, % natbib compatibility mode (\citep and \citet still work)
url=false, %
doi=false, %
eprint=false %
}{biblatex}
\usepackage{biblatex}

\addbibresource{library.bib}

\hypersetup{
% Uncomment the line below to remove all links (to references, figures, tables, etc), useful for b/w printouts
%draft,
colorlinks=true, linktocpage=true, pdfstartpage=3, pdfstartview=FitV,
% Uncomment the line below if you want to have black links (e.g. for printing black and white)
%colorlinks=false, linktocpage=false, pdfborder={0 0 0}, pdfstartpage=3, pdfstartview=FitV,
breaklinks=true, pdfpagemode=UseNone, pageanchor=true, pdfpagemode=UseOutlines,%
plainpages=false, bookmarksnumbered, bookmarksopen=true, bookmarksopenlevel=1,%
hypertexnames=true, pdfhighlight=/O,%nesting=true,%frenchlinks,%
urlcolor=webbrown, linkcolor=RoyalBlue, citecolor=Maroon, %pagecolor=RoyalBlue,%
%urlcolor=Black, linkcolor=Black, citecolor=Black, %pagecolor=Black,%
}

% My default font
%\usepackage{newcent}

% Computer Modern Bright font
%\usepackage{cmbright}

% Iwona light
\usepackage[light,math]{iwona}

% LX fonts
%\usepackage{lxfonts}

% Malvern
%\input T1fmv.fd
%\renewcommand*\sfdefault{fmv}
%\renewcommand*\familydefault{\sfdefault}

% Comfortaa
%\usepackage[default]{comfortaa}

%\setbeamercolor{frametitle}{fg=NavyBlue}
%\setbeamercolor{structure}{fg=NavyBlue}
%\setbeamercolor{normal text}{fg=black}
\setbeamercolor{alerted text}{fg=NavyBlue}
%\setbeamercolor{example text}{fg=red}

\newcommand{\cl}[1]{\multicolumn{1}{c}{#1}}

\newenvironment{changemargin}[2]{%
  \begin{list}{}{%
    \setlength{\topsep}{0pt}%
    \setlength{\leftmargin}{#1}%
    \setlength{\rightmargin}{#2}%
    \setlength{\listparindent}{\parindent}%
    \setlength{\itemindent}{\parindent}%
    \setlength{\parsep}{\parskip}%
  }%
\item[]}{\end{list}}

\begin{document}
\tikzstyle{every picture}+=[remember picture]
\tikzstyle{na} = [baseline=-.5ex]

\begin{frame}
\title{Formación y Evolución de las Galaxias}
%\subtitle{}
\author{Alfredo J. Mej\'ia$^{1,2}$}

\date{\today}

\institute{$^{1}$Posgrado de F\'isica Fundamental\\ Universidad de Los Andes \and $^{2}$Centro de %
Investigaciones de Astronom\'ia%
}

\maketitle
\end{frame}

%Motivación
\begin{frame}[allowframebreaks]{\textsc{Motivación}}
%
Si bien los sondeos de galaxias de última generación han revelado detalles de los fenómenos físicos
que intervienen en la formación y evolución de las galaxias, una construcción física \emph{ab
initio} permanece aún elusiva. Existen básicamente dos formas de modelar los procesos físicos de la
formación y la evolución de las galaxias: uno es mediante simulaciones hidrodinámicas de materia
oscura y bariónica y el otro es el llamado método semianalítico; aunque ambos métodos son
fundamentalmente distintos, obtienen resultados similares. Más aún, estos están en acuerdo
cualitativo con las observaciones. Esto nos dice que hemos llegado a comprender, al menos, los
fenómenos físicos fundamentales que dan forma al universo observado. Aún así, existen fenómenos
físicos de de gran importancia en la construcción de una teoría de formación de galaxias, que están
pobremente entendidos. Es el objeto de este seminario hacer un recuento fenomenológico de los
procesos físicos que intervienen en la formación de las galaxias, señalar las incertidumbres que
existe en cada proceso y establecer las perspectivas a futuro. Ya que el enfoque es fenomenológico,
este seminario estará naturalmente sesgado hacia los resultados del método semianalítico, sin
embargo, cuando sea pertienente señalaré las diferencias entre los resultados de ambos métodos en
comparación con las observaciones.
%
\end{frame}

%Historia
\begin{frame}[allowframebreaks]{\textsc{Antecedentes}}
%
\begin{itemize}
\item Primeros intentos por teorizar la formación de las galaxias.
\item ¿Cuáles eran los ingredientes físicos?.
\item ¿Qué lograban predecir?.
\end{itemize}
%
\end{frame}

\begin{frame}[allowframebreaks]{\textsc{Teoría de formación de galaxias}}
%
Presentar esquemáticamente:

\begin{itemize}
\item Fundamento físico.
\item Fenómenos que sabemos intervienen.
\item Resultados.
\end{itemize}
%
\end{frame}

%
\begin{frame}[allowframebreaks]{\textsc{Simulaciones hidrodinámicas}}
%
\begin{itemize}
\item Física de entrada.
\item Procedimientos.
\item Resultados.
\end{itemize}
%
\end{frame}

\begin{frame}[allowframebreaks]{\textsc{Simulaciones semianalíticas}}
%
\begin{itemize}
\item Física de entrada.
\item Procedimientos.
\item Resultados.
\end{itemize}
%
\end{frame}

\begin{frame}[allowframebreaks]{\textsc{Análisis comparativo}}
%
\begin{itemize}
\item Fortalezas y debilidades de ambos métodos.
\item Perspectivas.
\end{itemize}
%
\end{frame}

\begin{frame}[allowframebreaks]{\textsc{Poder predictivo}}
%
\begin{itemize}
\item Comparación de los resultados de las simulaciones con las observaciones.
\item Discutir las incertidumbres que intervienen cuando se falla en la predicción.
\item ¿Cómo se puede mejorar?.
\end{itemize}
%
\end{frame}

\begin{frame}[allowframebreaks]{\textsc{Resumen y perspectivas}}
%
\begin{itemize}
\item Mencionar los fenómenos físicos fundamentales que intervienen en la formación de las galaxias.
\item Mencionar las incertidumbres en cada uno de estos fenómenos.
\item Mencionar resultados más importantes de las simulaciones.
\item Decir cuales son las perspectivas.
\end{itemize}
%
\end{frame}

\begin{frame}[allowframebreaks]{\textsc{Referencias}}
\printbibliography
\end{frame}

\end{document}
